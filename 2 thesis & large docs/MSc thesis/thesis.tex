\documentclass[a4paper,12pt,twoside,openright]{UGthesis}
%\documentclass[a4paper,11pt,twoside]{UGthesis}
\usepackage{fancyheadings}
\usepackage[dvips]{graphicx}
%\usepackage{graphpap}
\usepackage{ifthen}
\usepackage{enumerate}
\usepackage{cite}
%\usepackage{amsmath}
%\usepackage{amssymb}
%\usepackage{calrsfs}
\usepackage[mathscr]{eucal}
\usepackage[errorshow]{tracefnt}
%\usepackage{makeidx}
%\usepackage{fancybox}
%\usepackage{psboxit}

%\include{psfig}

\usepackage{here}
\usepackage[hang,small,bf]{caption}

% bra and ket notation:
% Ex: $\bra{1}, \ket{0}, \bracket{\alpha}{\beta}$
\newcommand{\bra}[1]{\left<#1\right|}
\newcommand{\ket}[1]{\left|#1\right>}
\newcommand{\bracket}[2]{\left<\left.#1\right|#2\right>}

% macros for mathboxes... Q?:should we use \footnotesize
%instead of \small???
\newcommand{\inone}[2]{{\footnotesize \texttt{\emph{In[#1] :=}}} & {\footnotesize \texttt{\textbf{#2}}}}
\newcommand{\outone}[2]{{\footnotesize \texttt{\emph{Out[#1] :=}}} & {\footnotesize
    \texttt{#2}}}
\newcommand{\intwo}[1]{{\footnotesize \texttt{\textbf{#1}}}}
\newcommand{\outtwo}[1]{{\footnotesize \texttt{#1}}}
\newcommand{\bmb}{\vspace{0.6cm}
      \begin{tabular}{l}}
\newcommand{\emb}{\end{tabular}
      \vspace{0.4cm}}

% Ordinary nth order (\Od) and 1st order (\OD) derivative
% Ex: \Od{3}{f}{x}  \OD{f}{x}
\newcommand{\Od}[3]{\frac{d^{#1}#2}{d#3^{#1}}}
\newcommand{\OD}[2]{\frac{d#1}{d#2}}

% n/m fractions
% Ex: \nfrac{2}{3}
\newcommand{\nfrac}[2]{{}^{#1}\!\!/\!{}_{#2}}




%\input MHmacros

% Dirac notation
%\newcommand{\ket}[1]{| #1 \rangle}
%\newcommand{\bra}[1]{\langle #1 |}
%\newcommand{\bracket}[3]{\langle #1 |#2| #3 \rangle}
%\newcommand{\braket}[2]{\langle #1 | #2 \rangle}

\newcommand{\vA}{\mathcal{A}}
\newcommand{\vB}{\mathcal{B}}
\newcommand{\vF}{\ensuremath{\mathcal{F}}}
\newcommand{\vH}{\ensuremath{\mathcal{H}}}
\newcommand{\vR}{\ensuremath{\mathcal{R}}}
\newcommand{\ordo}{\ensuremath{\mathcal{O}}}
\newcommand{\pa}{\partial}
\newcommand{\pd}[1]{\frac{\partial}{\partial #1}}
\newcommand{\pdf}[2]{\frac{\partial #2}{\partial #1}}
\newcommand{\mi}{Minkowski}
\newcommand{\La}{Lagrangian}
\newcommand{\la}{\mathcal{L}}
\newcommand{\lax}{\mathscr{L}}
\newcommand{\slashed}[1]{#1 \;\!\!\!\!\!\slash}
\newcommand{\dslash}{\slashed{\partial}}
\newcommand{\yslash}{y\;\!\!\!\!\slash}
\newcommand{\fd}{four-dimensional}
\newcommand{\sd}{six-dimensional}
\newcommand{\matlab}{\textsc{Matlab}}
\newcommand{\beq}{\begin{equation}}
\newcommand{\eeq}{\end{equation}}
\newcommand{\beqa}{\begin{eqnarray}}
\newcommand{\eeqa}{\end{eqnarray}}
\newcommand{\bdm}{\begin{displaymath}}
\newcommand{\edm}{\end{displaymath}}

\newcommand\ffam{\sffamily}
\newcommand\fser{\bfseries}
\newcommand\fsh{\upshape}

\newcommand\blankpage{\thispagestyle{empty}\mbox{}\newpage}

\newcommand\rctr{\renewcommand{\theenumi}{\roman{enumi}}}
\newcommand\actr{\renewcommand{\theenumi}{\arabic{enumi}}}
\renewcommand{\theenumii}{\alph{enumii}}

\newcommand\nn{\nonumber}

\newcommand\benu{\begin{enumerate}}
\newcommand\eenu{\end{enumerate}}
\newcommand\bit{\begin{itemize}}
\newcommand\eit{\end{itemize}}

\newcommand{\ctxt}[2]{\put(#1){\makebox(0,0){#2}}}
\newcommand{\ltxt}[2]{\put(#1){\makebox(0,0)[l]{#2}}}

%%%%%%%%%%%%%%%%%%%%%%%%%%%%%%%%%%%%%%%%%%
%
% Labeling and refering
%
%%%%%%%%%%%%%%%%%%%%%%%%%%%%%%%%%%%%%%%%%%
\newcommand{\eqnlab}[1]{\label{eqn:#1}}
\newcommand{\figlab}[1]{\label{fig:#1}}
\newcommand{\tablab}[1]{\label{tab:#1}}
\newcommand{\eqnref}[1]{(\ref{eqn:#1})}
\newcommand{\figref}[1]{\ref{fig:#1}}
\newcommand{\tabref}[1]{\ref{tab:#1}}
\newcommand{\Eqnref}[1]{Eq.~(\ref{eqn:#1})}
\newcommand{\Figref}[1]{Fig.~\ref{fig:#1}}
\newcommand{\Tabref}[1]{Table~\ref{tab:#1}}
\newcommand{\Eqsref}[1]{Eqs.~(\ref{eqn:#1})}
\newcommand{\Figsref}[1]{Figs.~\ref{fig:#1}}
\newcommand{\Tabsref}[1]{Tables~\ref{tab:#1}}

\newcommand{\chlab}[1]{\label{ch:#1}}
\newcommand{\seclab}[1]{\label{sec:#1}}
\newcommand{\sseclab}[1]{\label{ssec:#1}}
\newcommand{\chref}[1]{\ref{ch:#1}}
\newcommand{\secref}[1]{\ref{sec:#1}}
\newcommand{\ssecref}[1]{\ref{ssec:#1}}
\newcommand{\Chref}[1]{Chapter~\ref{ch:#1}}
\newcommand{\Secref}[1]{Section~\ref{sec:#1}}
\newcommand{\Ssecref}[1]{Subsec.~\ref{ssec:#1}}
\newcommand{\Chsref}[1]{Chapter~\ref{ch:#1}}
\newcommand{\Secsref}[1]{Sections.~\ref{sec:#1}}
\newcommand{\Ssecsref}[1]{Subsecs.~\ref{ssec:#1}}

%%%%%%%%%%%%%%%%%%%%%%%%%%%%%%%%%%%%%%%%%%%%%%%%%%%%%%%%%%%%%%%%%%%%%%%%%%%%%%%

%
%       Document
%
%%%%%%%%%%%%%%%%%%%%%%%%%%%%%%%%%%%%%%%%%%%%%%%%%%%%%%%%%%%%%%%%%%%%%%%%%%%%%%%

\RequirePackage{hyperref}

\begin{document}

%%%%%%%%%%%%%%%%%%%%%%%%%%%%%%%%%%%%
%
% Cover
%
%%%%%%%%%%%%%%%%%%%%%%%%%%%%%%%%%%%%

%\include{cover}

%\blankpage

\pagenumbering{roman}
\setcounter{page}{1}

%%%%%%%%%%%%%%%%%%%%%%%%%%%%%%%%%%%%%
%
% Titelblad
%
%%%%%%%%%%%%%%%%%%%%%%%%%%%%%%%%%%%%%

\thispagestyle{empty}

%\vspace*{-1cm}
%\vspace*{4mm}

\begin{center}
  {\fsh\ffam\fser Thesis for the degree of Master of Science in
     Physics}
\end{center}


%\vspace*{0.5cm}

\begin{center}
{\upshape\sffamily\bfseries\huge HUBBARD MODELS} \\[4mm]
{\upshape\sffamily\bfseries\large States and Transformations}\\[4mm]
\end{center}

\vspace*{2mm}
\begin{center}
        \rule{110mm}{2pt}
\end{center}

\vspace*{4mm}
\begin{center}
  {\fsh\ffam\fser\Large Andreas Enqvist}\\
\end{center}
\vfill
\vspace{3 cm}
\begin{center}
%       \includegraphics[width=4cm]{/usr/local/lib/cthlogo.eps}
%\includegraphics[width=4cm]{GU_fin.eps}
\includegraphics[width=4cm]{gulogo.eps}
%       \epsffile{/home/tfe/gran/tex/figures/AvancezM70mm.eps}
%       \hspace*{2cm}
%       \includegraphics[width=4cm]{/usr/local/lib/gulogo.eps}
\end{center}

\vfill
\begin{center}
        {\ffam\fsh Institute for Theoretical Physics\\*[1mm]
        Chalmers University of Technology\\*[-.5mm]
        and\\*[-.5mm]
        G\"oteborg University\\*[2mm]
        \today}
\end{center}


%%%%%%%%%%%%%%%%%%%%%%%%%%%%%%%%%%%%
%
% Tryckort-sida
%
%%%%%%%%%%%%%%%%%%%%%%%%%%%%%%%%%%%%

\mbox{}\thispagestyle{empty}\newpage
\vspace*{165mm}

%{\ffam
%       {\center ISBN  ??-????-???-? \\ ISSN ????-???? \\}
%       \hspace*{20mm}
%       {\center Bibliotekets reproservice\\ G\"oteborg 1997 \\}
%}

\blankpage

%%%%%%%%%%%%%%%%%%%%%%%%%%%%%%%%%%%
%
% Abstract
%
%%%%%%%%%%%%%%%%%%%%%%%%%%%%%%%%%%%

\thispagestyle{empty}
\begin{center}
        {\ffam
        {\fser\Large HUBBARD MODELS} \\[4mm]
        {\fser\large States and Transformations}\\[6mm]
%       {\fsh\ffam\fser\Large Extra line headline}\\[4mm]
        {\normalsize Andreas Enqvist  \\
        Institute for Theoretical Physics \\
        Chalmers University of Technology and G\"oteborg University \\
        SE-412 96 G\"oteborg, Sweden} \\[7mm]}
\end{center}

\centerline{\ffam\fser Abstract}
\medskip
\normalsize
\noindent\input{abstract}

%\vspace{1 cm}
%\centerline{\ffam\fser Sammanfattning}
%\medskip
%\normalsize
%\noindent\input{samman}

\vfill

\newpage



%%%%%%%%%%%%%%%%%%%%%%%%%%%%%%%%%%%%%%%%%
%
% Acknowledgments
%
%%%%%%%%%%%%%%%%%%%%%%%%%%%%%%%%%%%%%%%%%
\thispagestyle{plain}
\vspace*{4cm}

\centerline{\ffam\fser\Large Acknowledgments}
\medskip
\smallskip

\normalsize
\noindent\input{acknow}

%%%%%%%%%%%%%%%%%%%%%%%%%%%%%%%%%%%%%%%%%%%%%
%
% Table of contents
%
%%%%%%%%%%%%%%%%%%%%%%%%%%%%%%%%%%%%%%%%%%%%%
%\pagestyle{plain}
\cleardoublepage
\tableofcontents
\pagestyle{empty}

\cleardoublepage
\pagestyle{fancy}
\renewcommand{\chaptermark}[1]{\markboth{Chapter \thechapter\ \ \ #1}{#1}}
\renewcommand{\sectionmark}[1]{\markright{\thesection\ \ #1}}
\lhead[\fancyplain{}{\sffamily\thepage}]%
  {\fancyplain{}{\sffamily\rightmark}}
\rhead[\fancyplain{}{\sffamily\leftmark}]%
  {\fancyplain{}{\sffamily\thepage}}
\cfoot{}
\setlength\headheight{14pt}

\rctr

\setcounter{page}{1}
\pagenumbering{arabic}

%%%%%%%%%%%%%%%%%%%%%%%%%%%%%%%%%%%%%%%%%%%%
%
% Text
%
%%%%%%%%%%%%%%%%%%%%%%%%%%%%%%%%%%%%%%%%%%%%
%\baselineskip=14.56pt

\include{introduction}

\include{spinhalf}

\include{spinthree}

\include{twoband}

\include{program}

\include{results}

\appendix
\pagestyle{plain}
\include{appendix}

%%%%%%%%%%%%%%%%%%%%%%%%%%%%%%%%%%%%%%%%%%%%
%
% Bibliography
%
%%%%%%%%%%%%%%%%%%%%%%%%%%%%%%%%%%%%%%%%%%%%
%\pagestyle{empty}
\cleardoublepage
%\clearpage
\pagestyle{plain}
\def\href#1#2{#2}
\bibliographystyle{utphysmod2}
%\bibliographystyle{plain}
\addcontentsline{toc}{chapter}{\sffamily\bfseries References}
\bibliography{biblio}
\clearpage 
\begin{thebibliography}{99}
\thispagestyle{fancy}

\bibitem{Fazekas} P. Fazekas, Lecture Notes on Electron Correlation and
  Magnetism, World Scientific Publishing Co., Singapore, 1999.
\bibitem{Mott}N.F. Mott, Metal-Insulator Transitions, Taylor \& Francis Ltd,
  London, 1974.
\bibitem{Hubbard} J. Hubbard, Proc. Roy. Soc. \textbf{A276} p238
  (1963), London. 
\bibitem{Gutzwiller} M.C. Gutzwiller, Phys Rev Lett. \textbf{10} p159 (1963).
\bibitem{Kanamori} Kanamori, Progr. Theor. Phys. \textbf{30} p275 (1963), Tokyo.
\bibitem{Fradkin} E. Fradkin Field Theories of Condensed Matter
  Systems, Addison-Wesley, Redwood City, 1991.
\bibitem{Tasaki} H. Tasaki, Progr. Theor. Phys. \textbf{99} (1998).
\bibitem{Mathematica} Wolfram Research, Inc., Mathematica, Version
  5.0, Champaign, IL (2003). 

\bibitem{Mele} S. \"Ostlund, E. Mele, Phys. Rev. B \textbf{44} 12413 (1991).
\bibitem{testet} S. \"Ostlund, T.H. Hanson, A. Karlhede, ArXiv
  Cond. matt. e-prints, cond-mat/0406717 (2004).
\bibitem{Zhang} C. Wu, J.P. Hu, S.C. Zhang,
  Phys. Rev. Lett. \textbf{91} p186402 (2003). 

\end{thebibliography}

%\bibitem{Fazekas} P. Fazekas, Lecture Notes on Electron Correlation and
%  Magnetism, World Scientific Publishing Co., Singapore, 1999.

%%%%%%%%%%%%%%%%%%%%%%%%%%%%%%%%%%%%%%%%%%%%
%
% Papers
%
%%%%%%%%%%%%%%%%%%%%%%%%%%%%%%%%%%%%%%%%%%%%

\end{document}


