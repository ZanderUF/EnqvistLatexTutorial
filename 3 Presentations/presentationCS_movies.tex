%%
%% This is file `powerdot-example1.tex',
%% generated with the docstrip utility.
%%
%% The original source files were:
%%
%% powerdot.dtx  (with options: `pdexample1')
%%
%% ---------------------------------------------------------------
%% Copyright (C) 2005-2006 Hendri Adriaens and Christopher Ellison
%% ---------------------------------------------------------------
%%
%% This work may be distributed and/or modified under the
%% conditions of the LaTeX Project Public License, either version 1.3
%% of this license or (at your option) any later version.
%% The latest version of this license is in
%%   http://www.latex-project.org/lppl.txt
%% and version 1.3 or later is part of all distributions of LaTeX
%% version 2003/12/01 or later.
%%
%% This work has the LPPL maintenance status "maintained".
%%
%% This Current Maintainer of this work is Hendri Adriaens.
%%
%% This work consists of all files listed in manifest.txt.
%%
\documentclass[clock,style=fyma,paper=screen,blackslide,trans=Wipe,mode=present]{powerdot}
\usepackage{amsmath}
\usepackage{movie15}

%\usepackage{pst-node}
%\usepackage{listings}
%\usepackage{babel}
%\usepackage[latin1]{inputenc}    % Accept european-encoded (latin1) characters.

\title{Noise Diagnostics of Stationary and Non-Stationary Reactor Processes}
\author{Carl Sunde}
\date{Chalmers University of Technology}
\pdsetup{lf= April 27 2007}

\begin{document}

\maketitle[{logohook=t, logopos={0.5\slidewidth,
0.47\slideheight},
logocmd={\includegraphics[width=5cm]{figures/framsida2.ps}}}]
%\begin{slide}[toc=,bm=]{Overview}
%\tableofcontents[content=sections]
%\end{slide}
%\section{Number Distributions}
\section[slide=false]{Erata}
\begin{slide}{Erata}
Figure 2.2 page 6: \quad \textit{dotted} should read \textit{dashed }\\[5mm]
In section 7.2 pages 50-51: \quad
$e^{\frac{|\textbf{r}-\textbf{r}'|}{l}}$ should read
$e^{\frac{|\textbf{r}'-\textbf{r}''|}{l}}$
\end{slide}

\section[slide=false]{Papers}

\begin{slide}{Part 1, wavelets analysis}
This thesis is an introduction to and a summary of the work
published in the following papers:
\begin{itemize}
\item
\textsc{Paper I}\\ C.~Sunde, S.~Avdi\'{c} and I.~P\'{a}zsit, "Classification of two-phase flow regimes via image analysis and a neuro-wavelet approach"\\
\emph{Progress in Nuclear Energy}, {\bf 46}, 348 (2005).
\item
\textsc{Paper II}\\C.~Sunde and I.~P\'{a}zsit, "Investigation of detector tube impacting in the Ringhals-1 BWR"\\
\emph{International Journal of Nuclear Energy Science and
Technology}, {\bf 2}, 189 (2006).
\item
\textsc{Paper III}\\ C.~Sunde and I.~P\'{a}zsit, "Wavelet techniques for the determination of the decay ratio in boiling water reactors"\\
\emph{Kerntechnik}, \textbf{72}, 7 (2007).
\end{itemize}
\end{slide}

\begin{slide}{Part 2, core barrel vibrations}
\begin{itemize}
\item
\textsc{Paper IV}\\C.~Sunde, C.~Demazi\`{e}re and I.~P\'{a}zsit , "Calculations of the neutron noise induced by shell-mode core-barrel vibrations in a 1-D, two-group, two-region slab reactor model"\\
\emph{Nuclear Technology}, {\bf 154}, 129 (2006).

\item
\textsc{Paper V}\\C.~Sunde, C.~Demazi\`{e}re and I.~P\'{a}zsit , "Investigation of the neutron noise induced by shell-mode core-barrel vibrations in a reflected reactor"\\
\emph{Proc. Int. Top. Mtg. on Mathematics and Computing,
Supercomputing, Reactor Physics and Nuclear and Biological
Applications (M\&C2005)}, September 12-15, 2005. \item
\textsc{Paper VI}\\M.~P\'{a}zsit, C.~Sunde and I.~P\'{a}zsit,
"Beam mode core-barrel vibrations in the PWRs Ringhals 2-4",
\emph{Proc. Int. Top. Mtg. on Advances in Nuclear Analysis and
Simulations (PHYSOR2006)}, September 10-14,
2006.
\end{itemize}
\end{slide}

\begin{slide}{Part 3, break-frequency method}
\begin{itemize}
\item \textsc{Paper VII}\\
C.~Sunde, C.~Demazi\`{e}re and I.~P\'{a}zsit, "Investigation of
the validity of the point-kinetics approximation and of the
break-frequency method in 2-D subcritical systems", \emph{Proc.
Joint Int. Top. Mtg. on Mathematics and Computing and
Supercomputing in Nuclear Applications (M\&C + SNA 2007)},
Monterey, California, April 15-19, 2007, American Nuclear Society
(2007)
\end{itemize}
\end{slide}

\section[slide=false]{Noise diagnostics}

\begin{slide}{Noise diagnostics}
\begin{itemize}
\item Noise is the time-dependent fluctuation from the mean
value of the neutron flux.\\[5mm]

\item Power reactor noise can be induced by technological
processes, vibrations of core components, or
temperature or density variations.\\[5mm]

\item Fluctuations of the cross-sections leads to fluctuations of
the neutron flux, noise.\\[5mm]

\item The noise can be used to identify deteriorating or
malfunctioning components, parameter estimation (e.g. reactivity,
moderator temperature coefficient and decay ratio)\\[5mm]
\end{itemize}
\end{slide}


\begin{slide}{Stationary and Non-Stationary}
\begin{itemize}
\item Stationary processes are uniform in time (frequencies do not
change)
\\ Beam-mode core barrel vibration or detector tube vibrations\\
[3mm]

\item Non-Stationary processes are intermittent or transient
\\ Fuel assembly vibrations due to detector impacting
\end{itemize}
\vspace{-5mm}
\begin{figure}[ht]
\includemovie[
  poster]{4.8cm}{5cm}{figures/sond2.avi}
\end{figure}
\end{slide}

\section[slide=false]{Detector tube vibrations}
\begin{slide}{Detector tube vibrations}
\begin{itemize}
\item Stationary and Non-stationary processes

\item Spectral and Wavelet analysis
\end{itemize}
\begin{figure}[H]
\centering
\includegraphics[width=7 cm]{figure_Lic/fuelbox2.eps}
\end{figure}

\end{slide}

\begin{slide}{Signals}
\begin{figure}[H]
\centering
\includegraphics[width=10 cm]{figure_Lic/vibration2.eps}
\end{figure}
\end{slide}

\begin{slide}{Spectral analysis}
\begin{itemize}
\item Broad vibration peak at the eigenfrequency

\item multiple peaks at double and triple frequencies

\item zero phase around the eigenfrequency

\item high coherence around the eigenfrequency
\end{itemize}
\begin{figure}[H]
\begin{tabular}{l l}
\hspace{-5mm}
\includegraphics[width=5 cm]{figure_Lic/cutNEWLPRM16.eps} &
\includegraphics[width=5 cm]{figures/LPRM2.eps} \\
\end{tabular}
\end{figure}
\end{slide}

\begin{slide}{Discrete wavelet analysis}
\begin{figure}[H]
\centering
\includegraphics[width=8 cm]{figures/discrete16.eps}
\end{figure}
\begin{equation}
x(t)=x_{M}(t)+ \sum_{n=1}^{n=M} d_n(t)
\end{equation}
\end{slide}

\begin{slide}{Continuous wavelet analysis}
\begin{figure}[H]
\centering
\includegraphics[width=8 cm]{figures/wavecoh.ps}
\end{figure}
\end{slide}

\begin{slide}{Results}
\begin{itemize}
\item Four measurements at BOC 2002-2005 from the BWR Ringhals-1
have been analysed

\item All with spectral and three with wavelet methods due to low
sampling frequency in the first measurement

\item four detector tubes are suspected for impacting in all or
all but one of the measurements

\item visual inspection in 2007 and analysis of measurement from
2006
\end{itemize}
\vspace{-5mm}
\begin{figure}[H]
\centering
\includegraphics[width=5 cm]{figures/detectormap.eps}
\end{figure}
\end{slide}

\section[slide=false]{Core-barrel vibrations}
\begin{slide}{Core-barrel vibrations}
Shell mode vibration, in-core noise used to classify the
vibration?

\begin{figure}[ht]
\includemovie[
  poster]{10cm}{5cm}{figures/corebarrel.avi}
\end{figure}
\end{slide}

\begin{slide}{Model of the vibrations}
\begin{itemize}
\item 1-D analytical calculation of the induced in-core noise
(2-group diffusion model)

\item 1-D numerical calculation of the induce in-core noise
(2-group diffusion model)

\item comparison with measurement

\begin{figure}[htb]
\centering
\includegraphics[width=9.5 cm]{figures/system.eps}\\
\end{figure}
\end{itemize}

\end{slide}

\begin{slide}{Analytical and numerical results}
\begin{figure}[htb]
\centering
\includegraphics[width=10 cm]{figures/AnaNum.eps}\\
\end{figure}
\end{slide}

\begin{slide}{Comparison with measurements}
\begin{figure}[htb]
\centering
\includegraphics[width=10 cm]{figures/AnaMeascolor.eps}\\
\end{figure}
\end{slide}

\section[slide=false]{General conclusions}
\begin{slide}{Conclusions}
\begin{itemize}
\item Important to have access to real measurements and data from
nuclear power plants\\[5mm]

\item In four of the five research areas the algorithms have been
tested on real measurements and data with success.\\[5mm]

\item Wavelets can be of use in some applications but they are not
a magic tool which solves all problems in signal processing.
\\[5mm]

\item Some of the methods and algorithms elaborated in this thesis are in routinely use for diagnosing operating power reactors\\[5mm]

\end{itemize}
\end{slide}


%\begin{slide}{Slide 2}
%  \begin{itemize}
%    \item<1-> Here
%    \begin{itemize}
%      \item<2-> we
%      \begin{itemize}
%        \item<3-> demonstrate
%        \begin{itemize}
%          \item<4-> the itemize environment
%        \end{itemize}
%      \end{itemize}
%    \end{itemize}
%  \end{itemize}
%\end{slide}

%\begin{slide}{Slide 3}
%  \begin{enumerate}[type=1]
%    \item<1> Here
%    \begin{enumerate}
%      \item<2> we
%      \begin{enumerate}
%        \item<3> demonstrate
%        \begin{enumerate}
%          \item<4> the enumerate environment
%        \end{enumerate}
%      \end{enumerate}
%    \end{enumerate}
%  \end{enumerate}
%\end{slide}




\end{document}
\endinput
%%
%% End of file `powerdot-example1.tex'.
