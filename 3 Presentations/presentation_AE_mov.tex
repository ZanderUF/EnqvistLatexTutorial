%%
%% This is file `powerdot-example1.tex',
%% generated with the docstrip utility.
%%
%% The original source files were:
%%
%% powerdot.dtx  (with options: `pdexample1')
%%
%% ---------------------------------------------------------------
%% Copyright (C) 2005-2006 Hendri Adriaens and Christopher Ellison
%% ---------------------------------------------------------------
%%
%% This work may be distributed and/or modified under the
%% conditions of the LaTeX Project Public License, either version 1.3
%% of this license or (at your option) any later version.
%% The latest version of this license is in
%%   http://www.latex-project.org/lppl.txt
%% and version 1.3 or later is part of all distributions of LaTeX
%% version 2003/12/01 or later.
%%
%% This work has the LPPL maintenance status "maintained".
%%
%% This Current Maintainer of this work is Hendri Adriaens.
%%
%% This work consists of all files listed in manifest.txt.
%%
\documentclass[clock,style=horatio,paper=screen,blackslide,trans=Wipe,mode=present]{powerdot}
\usepackage{amsmath}
\usepackage{subfigure}
\usepackage{movie15}
%\usepackage{pst-node}
%\usepackage{listings}
%\usepackage{babel}
%\usepackage[latin1]{inputenc}    % Accept european-encoded (latin1) characters.
\vspace{-2cm}
\title{\vspace{-1cm}Safeguards: Modelling~of~the~Detection and Characterization of Nuclear Materials  \vspace{-0.3cm}}
\author{  Andreas Enqvist \\ Chalmers University of Technology}
\date{}
\pdsetup{lf={February 26, 2010}}

\begin{document}

\maketitle[{logohook=t, logopos={0.5\slidewidth,
0.55\slideheight},
logocmd={\includegraphics[width=6cm]{figures/frontpage.eps}}}]
%\begin{slide}[toc=,bm=]{Overview}
%\tableofcontents[content=sections]

\section[slide=false]{Corrigenda}
\begin{slide}{Corrigenda}
Status of \textsc{Paper IV} should now read \emph{Nuclear
Instruments \& Methods $\boldsymbol{A}$}, \textbf{615}, 62--69
(2010). \\[4mm]

Status of \textsc{Paper VII} should now read \emph{accepted for
publication in Nuclear Instruments \& Methods $\boldsymbol{A}$},
(2010). \\ DOI:   10.1016/j.nima.2010.02.119
\end{slide}

\section[slide=false]{Papers}

\begin{slide}{Part 1, Particle Number Distribution}
This thesis is an introduction to and a summary of the work
published in the following papers:
\begin{itemize}
\item \textsc{Paper I}\\[1mm] A.~Enqvist, I.~P\'{a}zsit and S.A.~Pozzi, ``The Number Distribution of Neutrons and Gamma Photons Generated in a Multiplying Sample'' \\
\emph{Nuclear Instruments \& Methods $\boldsymbol{A}$}, {\bf
566}, 598--608 (2006).

\item \textsc{Paper II}\\[1mm] A.~Enqvist, I.~P\'{a}zsit and S.A.~Pozzi, ``The Detection Statistics of Neutrons and Photons Emitted from a Fissile Sample''\\
\emph{Nuclear Instruments \& Methods $\boldsymbol{A}$},
\textbf{607}, 451--457 (2009).
\end{itemize}
\end{slide}


\begin{slide}{Part 2, Multiplicity Theory}
\begin{itemize}
\item \textsc{Paper III}\\[1mm] I.~P\'{a}zsit, A.~Enqvist and L.~P\'{a}l, ``A note on the multiplicity expressions in nuclear safeguards.''\\
\emph{Nuclear Instruments \& Methods $\boldsymbol{A}$},
\textbf{603}, 541--544 (2009).

\item \textsc{Paper IV}\\[1mm] A.~Enqvist, I.~P\'{a}zsit and S.~Avdic,``Sample Characterization Using Both Neutron and Gamma Multiplicities''\\
\emph{Nuclear Instruments \& Methods $\boldsymbol{A}$},
\textbf{615}, 62--69 (2010).

\item \textsc{Paper V}\\[1mm] S.~Avdic, A.~Enqvist and I.~P\'{a}zsit, ``Unfolding sample parameters from neutron and gamma multiplicities using artificial neural networks.''\\
\emph{ESARDA Bulletin}, \textbf{43}, 21--29 (2009). Invited
paper.
\end{itemize}
\end{slide}


\begin{wideslide}{Part 3-4, Scintillation Detectors \& Correlation Measurements}
\small \begin{itemize}
\item \textsc{Paper VI}\\[1mm] S.A.~Pozzi, M.~Flaska, A.~Enqvist and I.~P\'{a}zsit, ``Monte Carlo and Analytical Models of Neutron Detection with Organic Scintillation Detectors.''\\
\emph{Nuclear Instruments \& Methods $\boldsymbol{A}$},
\textbf{582}, 629--637 (2007).

\item \textsc{Paper VII}\\[1mm] A.~Enqvist and I.~P\'{a}zsit, ``Calculation of the light pulse distributions induced by fast neutrons in organic scintillation detectors.''\\
\emph{Accepted for publication in Nuclear Instruments \& Methods
$\boldsymbol{A}$}.

\item \textsc{Paper VIII}\\[1mm] A.~Enqvist, M.~Flaska and S.A.~Pozzi, ``Measurement of Neutron/gamma-ray Cross-Correlation Functions for the Identification of Nuclear Materials''\\
\emph{Nuclear Instruments \& Methods $\boldsymbol{A}$},
\textbf{595}, 426--430 (2008).

\item \textsc{Paper IX}\\[1mm] A.~Enqvist, M.~Flaska and S.A.~Pozzi, ``Initial Evaluation for a Combined Neutron and Gamma-ray Multiplicity Counter''\\
\emph{Submitted to Nuclear Instruments \& Methods
$\boldsymbol{A}$}.
\end{itemize}
\end{wideslide}


\section[slide=false]{Nuclear Safeguards}

\begin{slide}{Safeguards}
\begin{itemize}
    \item Safeguards is a very broad area ranging from nuclear
        physics to political regulations.

    \item Three main tools are used for describing the physics
        of nuclear materials:
        \begin{itemize}
            \item Analytical models.
            \item Simulations.
            \item Experiments.
        \end{itemize}
    \item All of which have strengths and weaknesses.

    \item The work presented in the thesis is mostly based on
        analytical methods for describing the physical
        processes governing the multiplication and detection
        of neutrons and gamma rays. The results have often
        been compared to simulations, and experiments in form
        of measurements have also been performed to gain
        additional insight.
\end{itemize}
\end{slide}


\begin{slide}[toc=]{Potential Perils?}
The motivation of the work is the double-edged sword that nuclear
materials presents: It is a way to provide carbon emission-free
energy, while at the same time producing waste which needs careful
management.
\begin{figure}[H]
\centering
\includegraphics[width=6.5 cm]{Figures/nuclear_potential.eps}
\end{figure}
\end{slide}

\begin{slide}[toc=]{The Task}
\vspace{6mm}
\begin{itemize}
    \item The subject of the thesis is the non-intrusive investigation of nuclear materials. \\[6mm]

    \item Detection, identification and quantification of
        nuclear materials from the properties of the detected
        radiation.
\end{itemize}
\end{slide}

\section[slide=false]{Number Distributions}

\begin{slide}{Branching Processes and Fissile Material}
In fissile material the life of neutrons is a branching process in
which neutrons can generate additional neutrons and reactions.
\begin{itemize}
    \item Correlated events.
    \item Microscopic physics generating macroscopic effects.
\end{itemize} \vspace{-3mm}
\centering
\begin{figure}[ht]
\includemovie[
  poster]{5cm}{5cm}{figures/example.avi}
\end{figure}

%\includegraphics[width=6.5 cm]{Figures/branch4.eps}
\end{slide}


\begin{slide}{Probability Distributions}
\vspace{-1mm}
\begin{itemize}
\item Neutrons in a fissile sample can undergo a number of
    different processes.
\begin{itemize}
\item Fission.

\item Capture.

\item $(n,x n)$-reactions.

 \item Escape the sample.
\end{itemize}
\item Using the associated reaction probabilities, which depend on
    sample characteristics, probability balance equations can be
    formulated.

\item Generating functions and master equations then provide the
    tools needed to describe the behaviour of particles in the
    sample.

\item The goal: the probability $p(n)$ of generating $n$
    particles when starting with a single particle or a source
    event such as spontaneous fission with the emission of
    several neutrons.
\end{itemize}
\end{slide}


\begin{wideslide}[toc=]{The Theory}
\vspace{-2mm}
\begin{equation} p_1(n)=(1-\mathrm{p})\delta_{n,1} +
\mathrm{p} \sum_{k=1}^{\infty} p_i(k)
\begin{array}{c}  \quad \quad \quad \quad \quad \sum \quad \quad \quad  \prod^k_{i=1} p_1(n_i) \cr
\{n_1+n_2+ \ldots + n_k=n\} \quad \quad \quad  \end{array}.
\end{equation}
converted using generating functions to simple master equations:
\vspace{-2mm}
\begin{eqnarray}
\label{eq:hz}
h(z)&=&(1-\mathrm{p})z+ \mathrm{p} q_i[h(z)], \\
H(z)&=&q_{sf}[h(z)].
\end{eqnarray}
giving probability terms such as: \vspace{-2mm}
\begin{equation}
 P(2)= \frac{1}{2} \left(\frac{1-p}{1-p\overline{\nu_f}}\right)^{2}
 \left[ \overline{\nu_s(\nu_s-1)} +
 \frac{p}{1-p\overline{\nu_f}} \overline{\nu_s} \overline{\nu_f(\nu_f-1)}\right].
\end{equation}\vspace{-2mm}
through:\vspace{-2mm}
\begin{equation}
\label{eq:probp} p_1(n) = \left.\frac{1}{n!}
\frac{d^{n}h(z)}{dz^{n}} \right|_{z=0} \quad \mbox{and} \quad P(n)
= \left.\frac{1}{n!} \frac{d^{n}H(z)}{dz^{n}} \right|_{z=0}.
\end{equation}
\end{wideslide}


\begin{wideslide}{Neutrons}
\twocolumn[topsep=1cm]{
\includegraphics[width=6.7cm]{Figures/probspontneutBIG2.eps}}
{\begin{itemize}
\item Probabilities of having $n$ neutrons generated when starting
    with a source event, as calculated from expressions derived
    from master equations.

\item Dependence on mass shown for 20 wt\% $^{240}$Pu and 80
    wt\% $^{239}$Pu  metallic samples.

\item Change compared to non-multiplying case.

\item Excellent agreement with MCNP-PoliMi.

%\item Model was extended also to include absorption and detection.
\end{itemize}}
\end{wideslide}


\begin{slide}{Gamma Rays}
Gamma rays have no self multiplication, instead they are
    generated as a by-product of the branching process of
    neutrons.
\twocolumn[topsep=0.5cm]{\vspace{-3mm}
\includegraphics[width=5.8cm]{Figures/gamma45BIG.eps}}
{\vspace{1mm}\begin{itemize}
\item Dependence on mass shown.  \vspace{2mm}
\item Wider distribution compared to neutrons. \vspace{2mm}
\item Contains the information needed for calculating the moments.
\end{itemize}}\vspace{3mm}
The models were extended to also include absorption and
    detection.
\end{slide}



\section[slide=false]{Multiplicity Theory}

\begin{slide}{Multiplicity Counting}
\begin{itemize}
    \item Neutron multiplicity counting is used extensively
        for materials control and accountability (MC\&A). \\[2mm]
    \item The system has four parameters: neutron leakage
        multiplication, alpha ratio, fission rate and neutron
        detection efficiency.  \\[2mm]
    \item Three multiplicity rates are normally measured:
        singles, doubles and triples. \\[2mm]
    \item The systems are usually based on $^{3}$He-detectors,
        which are sensitive to slow neutrons. \\[2mm]
    \item The analysis is based on the same type of master
        equations previously mentioned. \\[2mm]
    \item Use of scintillation detectors opens of
        possibilities of detecting not only neutrons.
\end{itemize}
\end{slide}


\begin{slide}{Extensions and Neural Networks}
\begin{itemize}
    \item The formalism was extended to take into account not
        only gamma rays, but also mixed particle multiples
        such as $nn\gamma$, $n\gamma$. \\[2mm]
    \item The generation of neutrons and gamma rays shows
        interdependence. \\[2mm]
    \item The analysis grows more and more complicated which
        motivated the usage of artificial neural networks for inverting the solutions (unfolding). \\[2mm]
    \item Parameter unfolding using the information in an
        over-determined system. \\[2mm]
\end{itemize}
\vspace{-1mm} \centering
\includegraphics[width=9.0cm]{Figures/ANN3.eps}
\end{slide}


\begin{wideslide}{Method Validation}

\begin{table}[!hbt]%[H]  %place at bottom of section?
\centering
\begin{tabular}{c||c|c|c}
  % after \\: \hline or \cline{col1-col2} \cline{col3-col4} ...
  & fission rate (F) & $\alpha$  &  $\mathrm{p}$   \\     \hline \hline
  max. abs. rel. error (\%) & 0.0001 & 0.0021  & 0.0001  \\  \hline
  mean error of training (\%) & -1.98e-9 & -4.14e-7  & 4.96e-9  \\  \hline
  standard deviation of training (\%) & 7.56e-6 & 2.11e-4  & 3.24e-5  \\   \hline
  mean error of test data (\%) & 1.28e-6 & -7.15e-6  & 2.56e-6  \\  \hline
  standard deviation of test data (\%) & 9.34e-6 & 1.42e-4  & 3.29e-5  \\
\end{tabular}
\end{table}
\twocolumn[topsep=0cm]{\centering \vspace{-3mm}
\includegraphics[width=5.5cm]{Figures/neutron_input.eps}}
{\vspace{3mm}\begin{itemize}
\item The neural network was validated using neutron
    multiplicities.
\vspace{3mm}
\item The accuracy of the unfolded parameters, especially the
    fission rate (F), which is directly linked to the sample mass, is very encouraging.
\end{itemize}}
\end{wideslide}


\begin{wideslide}{ANN Results}
\begin{itemize}
    \item The multiples up to third order give a total of 9
        measurables. \\[2mm]
    \item Additional parameters such as gamma ray detection
        efficiency, gamma ratio and gamma leakage
        multiplicity, needs to be unfolded.
\end{itemize}
\twocolumn[topsep=0cm]{\centering
\includegraphics[width=6.5cm]{Figures/hist_errors6.eps}}
{\vspace{5mm}\begin{itemize}
\item Sensitivity analysis performed by adding noise to the input
    data.
    \vspace{2mm}
\item Omission of input parameters showed which measurables are
    most important and which could be omitted in a measurement due to redundancy.
\end{itemize}}
\end{wideslide}



\section[slide=false]{Scintillation Light Pulses}


\begin{slide}{Neutron Scattering}
\begin{itemize}
    \item The light pulses generated in a scintillation
        detector by fast neutrons, depend on the scattering
        history. \\[2mm]
    \item Energy is transferred in the collisions and
        transformed into light depending on the nuclei
        involved in the scattering. \\[2mm]
    \item The order and type of nuclei are very important to
        the generated light pulse. \\[6mm]
\end{itemize}
\centering
\includegraphics[width=8.5 cm]{Figures/collision2.eps}
\end{slide}


%\begin{wideslide}{Convolution Integrals}
%New types of mathematical tools needed. It is not a branching
%process but rather interdependent processes. \small \vspace{-1mm}
%\[
%    P_c = 1 - \frac{1}{h} \left[ E_3(0)-E_3(h) \right] +
%    + \frac{4}{\pi d^{2}h}
%    \int^{d}_0 dt \left\{ E_3(t)- E_3 \left[ (t^{2}+h^{2})^{1/2} \right]
%    \right\}(d^{2}+t^{2})^{1/2}
%\]
%\begin{equation}\label{eq:cyl}
%    - \frac{4}{\pi d^{2}h}
%    \int^{h}_0 du (h-u) \int^{d}_0 t^{2} dt  \frac{e^{\left[ -(t^{2}+u^{2})^{1/2}
%    \right]}}{(t^{2}+u^{2})^{3/2}} (d^{2} - t^{2})^{1/2} \equiv P(E),
%\end{equation} \vspace{2mm}
%
%\normalsize Cross section and collision history dependence.
%\small
%\begin{equation}
%\begin{array}{c}
%\displaystyle  f_{HCH}(L,E_0) = \int^{L-l_{1}}_{0} \int^{L}_{0} f_{H} \left[ L- (l_{1} + l_{2}), E_{0} - (T_{h}(l_{1}) + T_{c}(l_{2})) \right] \times \cr
%\cr
%\vspace{-2mm}
%   f_{C} \left[ l_{2}, E_{0} - T_{h}(l_{1}) \right] f_H [l_{1}, E_0] \times
%   W\left[ E_0 - T_h(l_{1}) - T_c(l_{2}) - T_h(L-(l_{1} + l_{2})) \right] dl_1 dl_2.
%\end{array}
%\end{equation}
%\end{wideslide}


\begin{wideslide}[toc=]{Convolution Integrals}
New types of mathematical tools needed. It is not a branching
process but rather interdependent processes: \small \vspace{-1mm}
\[
\begin{array}{c}
\displaystyle  f_{HCH}(L,E_0) = \int^{L-l_{1}}_{0} \int^{L}_{0} f_{H} \left[ L- (l_{1} + l_{2}), E_{0} - (T_{h}(l_{1}) + T_{c}(l_{2})) \right] \times \cr
\cr
\vspace{-2mm}
   f_{C} \left[ l_{2}, E_{0} - T_{h}(l_{1}) \right] f_H [l_{1}, E_0] \times
   W\left[ E_0 - T_h(l_{1}) - T_c(l_{2}) - T_h(L-(l_{1} + l_{2})) \right] dl_1 dl_2.
\end{array}
\]
\normalsize using: \small \vspace{2mm}
\[
f_{1h}(L,E_0 ) = \frac{1}{{E_0 \sqrt {b^2  +
4aL} }},
\]
\[
f_{1c}(L,E_0 ) = \frac{\theta (L_{max,c} -
L)}{{c\,(1 - \alpha)\, E_0}}.
\]
\[
T_h(L) = \frac{{\sqrt {b^2  + 4aL}  -
b}}{{2a}} \quad, \quad T_c(L) = L/c
\]
\end{wideslide}


\begin{wideslide}{Collision Histories}
The light pulse amplitude distribution was calculated for
individual collisions histories for mono-energetic neutrons of 1.5
MeV.

\twocolumn[topsep=0.2cm]{\centering \vspace{3mm}
\includegraphics[width=6.6 cm]{Figures/tripel15.eps}}
{\vspace{5mm}\begin{itemize}
\item Even  for collisions histories with the same type of nuclei,
    the light pulse distribution is very different. \\[1mm]
\item Comparisons with MCNP-PoliMi show good agreement. \\[1mm]
\item The results from individual collisions histories can be used
    to understand the shape of the full light pulse distribution.
\end{itemize}}
\end{wideslide}


\begin{slide}{Size Dependence}
The differences depending on detector size and collision history
are accurately highlighted.
\begin{figure}[hbt]
\centering
\subfigure[CCH, 1 cm.]{\label{fig:CCH1}
\includegraphics[width=3.6cm]{Figures/CCH1.eps}}
\subfigure[CCH, 10 cm.]{\label{fig:CCH10}
\includegraphics[width=3.6cm]{Figures/CCH10.eps}}
\subfigure[HCH, 1 cm.]{\label{fig:HCH1}
\includegraphics[width=3.6cm]{Figures/HCH1.eps}}
\subfigure[HCH, 10 cm.]{\label{fig:HCH10}
\includegraphics[width=3.6cm]{Figures/HCH10.eps}}
\end{figure}
\end{slide}


\section[slide=false]{Correlation Measurements}

\begin{slide}{Event Correlations}

\begin{itemize}
    \item Fissions generate a number of neutrons and gamma
        rays in each event.
    \item Simultaneously detecting multiple particles is
        therefore a good indicator of fissile material.
    \item Correlation measurements could also give additional
        information about sample parameters in a similar way
        to multiplicity measurements.
    \item Detecting both types of radiation has good benefits.
\end{itemize}
\centering
\begin{tabular}{r l}
\includegraphics[width=0.40\textwidth]{Figures/SNC00110.eps}   &
\includegraphics[width=0.45\textwidth]{Figures/particle2_2.eps}
\end{tabular}
\end{slide}


\begin{slide}{Cross-Correlations}
\begin{itemize}
    \item Cross-correlation (CC) measurements utilize two
        correlated pulses to give a sample-geometry signature. \\[1mm]
    \item It requires very fast detectors and systems, due to
        the inherent speed of the gamma rays and neutrons. \\[1mm]
    \item The accuracy of the CC functions depends on the
        pulse shape discrimination (PSD) and other
        considerations.
\end{itemize}
\vspace{-1mm} \centering
\begin{tabular}{r l}
\hspace{-4mm}
\includegraphics[width=0.52\textwidth]{Figures/CFDtotvsMC30.eps} &    \hspace{-5mm}
\includegraphics[width=0.52\textwidth]{Figures/CFDpartvsMC30.eps}
\end{tabular}
\end{slide}


\begin{slide}{Mixed Multiplicity Counting}
\begin{itemize}
    \item As seen from multiplicity theory, successful
        detection of multiples of both neutron and photons
        might be used to improve pure neutron multiplicity
        counting.
\end{itemize}
+ No moderation needed, more measurables. \\[1mm]

\, -- \,Low detection efficiency, requires PSD or two types of
detectors.

\centering
\includegraphics[width=6 cm]{Figures/rates3.eps}
\end{slide}



\section[slide=false]{General Conclusions}

\begin{slide}{Conclusions}
\begin{itemize}
    \item The work has shown the importance of using multiple
        approaches: models, simulations and experiments all
        have strengths and weaknesses. \\[5mm]
    \item Simple physical processes such as neutron scattering
        gives very complex analyzes and behaviours in for
        example scintillation detectors. \\[5mm]
    \item Performing measurements have allowed for a
        possibility to determine which ideas are applicable,
        and which are subject to technical limitations. \\[5mm]
    \item Algorithms and tools of mathematical physics used
        here occur not only in the area of safeguards but also
        other fields such as reactor physics.
\end{itemize}
\end{slide}



\end{document}
\endinput
