\documentclass[landscape,a0,final,a4resizeable]{a0poster}
%\documentclass[landscape,a0,final]{a0poster}
%\documentclass[portrait,a0b,final,a4resizeable]{a0poster}
%\documentclass[portrait,a0b,final]{a0poster}
%%% Option "a4resizeable" makes it possible ot resize the
%   poster by the command: psresize -pa4 poster.ps poster-a4.ps
%   For final printing, please remove option "a4resizeable" !!

\usepackage{epsfig}
\usepackage{multicol}
\usepackage{pstricks,pst-grad}



%%%%%%%%%%%%%%%%%%%%%%%%%%%%%%%%%%%%%%%%%%%
% Definition of some variables and colors
%\renewcommand{\rho}{\varrho}
%\renewcommand{\phi}{\varphi}
\setlength{\columnsep}{3cm}
\setlength{\columnseprule}{2mm}
\setlength{\parindent}{0.0cm}



%%%%%%%%%%%%%%%%%%%%%%%%%%%%%%%%%%%%%%%%%%%%%%%%%%%%
%%%               Background                     %%%
%%%%%%%%%%%%%%%%%%%%%%%%%%%%%%%%%%%%%%%%%%%%%%%%%%%%

\newcommand{\background}[3]{
  \newrgbcolor{cgradbegin}{#1}
  \newrgbcolor{cgradend}{#2}
  \psframe[fillstyle=gradient,gradend=cgradend,
  gradbegin=cgradbegin,gradmidpoint=#3](0.,0.)(1.\textwidth,-1.\textheight)
}



%%%%%%%%%%%%%%%%%%%%%%%%%%%%%%%%%%%%%%%%%%%%%%%%%%%%
%%%                Poster                        %%%
%%%%%%%%%%%%%%%%%%%%%%%%%%%%%%%%%%%%%%%%%%%%%%%%%%%%

\newenvironment{poster}{
  \begin{center}
  \begin{minipage}[c]{0.98\textwidth}
}{
  \end{minipage}
  \end{center}
}



%%%%%%%%%%%%%%%%%%%%%%%%%%%%%%%%%%%%%%%%%%%%%%%%%%%%
%%%                pcolumn                       %%%
%%%%%%%%%%%%%%%%%%%%%%%%%%%%%%%%%%%%%%%%%%%%%%%%%%%%

\newenvironment{pcolumn}[1]{
  \begin{minipage}{#1\textwidth}
  \begin{center}
}{
  \end{center}
  \end{minipage}
}



%%%%%%%%%%%%%%%%%%%%%%%%%%%%%%%%%%%%%%%%%%%%%%%%%%%%
%%%                pbox                          %%%
%%%%%%%%%%%%%%%%%%%%%%%%%%%%%%%%%%%%%%%%%%%%%%%%%%%%

\newrgbcolor{lcolor}{0. 0. 0.80}
\newrgbcolor{gcolor1}{1. 1. 1.}
\newrgbcolor{gcolor2}{.80 .80 1.}

\newcommand{\pbox}[4]{
\psshadowbox[#3]{
\begin{minipage}[t][#2][t]{#1}
#4
\end{minipage}
}}



%%%%%%%%%%%%%%%%%%%%%%%%%%%%%%%%%%%%%%%%%%%%%%%%%%%%
%%%                myfig                         %%%
%%%%%%%%%%%%%%%%%%%%%%%%%%%%%%%%%%%%%%%%%%%%%%%%%%%%
% \myfig - replacement for \figure
% necessary, since in multicol-environment
% \figure won't work

\newcommand{\myfig}[3][0]{
\begin{center}
  \vspace{1.5cm}
  \includegraphics[width=#3\hsize,angle=#1]{#2}
  \nobreak\medskip
\end{center}}



%%%%%%%%%%%%%%%%%%%%%%%%%%%%%%%%%%%%%%%%%%%%%%%%%%%%
%%%                mycaption                     %%%
%%%%%%%%%%%%%%%%%%%%%%%%%%%%%%%%%%%%%%%%%%%%%%%%%%%%
% \mycaption - replacement for \caption
% necessary, since in multicol-environment \figure and
% therefore \caption won't work

%\newcounter{figure}
\setcounter{figure}{1}
\newcommand{\mycaption}[1]{
  \vspace{0.5cm}
  \begin{quote}
    {{\sc Figure} \arabic{figure}: #1}
  \end{quote}
  \vspace{1cm}
  \stepcounter{figure}
}



%%%%%%%%%%%%%%%%%%%%%%%%%%%%%%%%%%%%%%%%%%%%%%%%%%%%%%%%%%%%%%%%%%%%%%
%%% Begin of Document
%%%%%%%%%%%%%%%%%%%%%%%%%%%%%%%%%%%%%%%%%%%%%%%%%%%%%%%%%%%%%%%%%%%%%%

\begin{document}

\background{1. 1. 1.}{1. 1. 1.}{0.5}

\vspace*{2cm}


\newrgbcolor{lightblue}{0. 0. 0.80}
\newrgbcolor{white}{1. 1. 1.}
\newrgbcolor{whiteblue}{.80 .80 1.}

\begin{poster}

%%%%%%%%%%%%%%%%%%%%%
%%% Header
%%%%%%%%%%%%%%%%%%%%%
\begin{center}
\begin{pcolumn}{0.98}

\pbox{0.95\textwidth}{}{linewidth=2mm,framearc=0.3,linecolor=lightblue,fillstyle=gradient,gradangle=0,gradbegin=white,gradend=whiteblue,gradmidpoint=1.0,framesep=1em}{

%%% Unisiegel
\begin{minipage}[c][9cm][c]{0.1\textwidth}
  \begin{center}
    \includegraphics[width=7cm,angle=0]{bild70mm.ps}
  \end{center}
\end{minipage}
%%% Titel
\begin{minipage}[c][9cm][c]{0.78\textwidth}
  \begin{center}
    {\sc \Huge Statistics of the neutrons and gamma photons emitted from a fissile \\[6mm]
    sample with absorption}\\[10mm]
    {\Large A. Enqvist$^{a}$, I. P\'azsit$^{a}$ and S. A. Pozzi$^{b}$\\[7.5mm]
    $^{a}$Department of Nuclear Engineering, Chalmers University of Technology, G\"oteborg,
    Sweden, $^{b}$Oak Ridge National Laboratory, Oak Ridge TN, USA}
%    Institute of Poster--Design, University, Your City, Country}
  \end{center}
\end{minipage}
%%% GK-Logo
\begin{minipage}[c][9cm][c]{0.1\textwidth}
  \begin{center}
    \includegraphics[width=11cm,angle=0]{Graphictest.eps}
  \end{center}
\end{minipage}

}
\end{pcolumn}
\end{center}


\vspace*{2cm}



%%%%%%%%%%%%%%%%%%%%%
%%% Content
%%%%%%%%%%%%%%%%%%%%%


%%% Begin of Multicols-Enviroment
\begin{multicols}{4}

%\small %\bf
%%% Abstract
\begin{center}\pbox{0.8\columnwidth}{}{linewidth=2mm,framearc=0.1,linecolor=lightblue,fillstyle=gradient,gradangle=0,gradbegin=white,gradend=whiteblue,gradmidpoint=1.0,framesep=1em}{\begin{center}\Large \bf Introduction\end{center}}\end{center}
\vspace{1.25cm}

This work investigates an analytical derivation of the
distribution of the number of neutrons and photons emitted by a
multiplying sample. The relationship between the statistics of the
generated and detected neutrons and photons is also described. The
analytical model described in this paper accounts for absorption
and detection, thus extending the model presented in previous
studies. By using this new, improved model, one can investigate
the relative feasibilities of measuring neutrons or gamma photons
for the analysis of a specific fissile sample. In fact, larger
mass will lead to larger self-shielding for gamma photons, whereas
for neutrons a larger mass will lead to increased multiplicities
due to an increased probability to induce fission for each
neutron, with absorption playing a minor role. The results suggest
that although photons have a larger initial (source)
multiplication, neutrons might be more favourable to measure in
the case of large samples because of the increasing self-shielding
effect for gamma photons.

%For fissile samples the heavy isotopes will have a large
%absorption of photons that increase with sample mass. This
%self-shielding effect will lower the number of emitted photons
%from the sample so drastically that for more massive samples it
%becomes favourable to examine samples using neutrons instead of
%photons in coincidence and multiplicity measurements


%%% Introduction
\vspace{2cm}\begin{center}\pbox{0.8\columnwidth}{}{linewidth=2mm,framearc=0.1,linecolor=lightblue,fillstyle=gradient,gradangle=0,gradbegin=white,gradend=whiteblue,gradmidpoint=1.0,framesep=1em}{\begin{center}\Large
\bf Theoretical Treatment of Neutrons
\end{center}}\end{center}\vspace{1.25cm}

In non-destructive assay of nuclear material the statistics of the
number distribution of neutrons and gamma rays emitted by fissile
samples play an important role. For fissile samples the key
processes regarding the distributions are: \vspace{5mm}
\begin{quote}
\begin{itemize}
    \item Spontaneous fission. \par

    \item Induced fission.

    \item Change of multiplicities due to:
    \begin{itemize}

    \item Induced fission.

    \item Absorption.

    \item The process of detection
\end{itemize}

    \item Multiplicities and coincidences can give isotopic composition and mass of the sample.
\end{itemize}
\end{quote}
\vspace{5mm} With the help of master equations one can write down
relationships describing the generating functions of the number
distribution. For a model including absorption but not detection
the equations read as:
\begin{equation}
\label{eq:hz} h(z)=(1-p')z+ p' \widetilde{q}_f[h(z)],
\end{equation}
\begin{equation}
\label{eq:Hz} H(z)=q_s[h(z)].
\end{equation}
Here the number distribution is taken for one initial neutron or
one initial neutron event:
\begin{equation}
\label{eq:probp} p_1(n) = \left.\frac{1}{n!}
\frac{d^{n}h(z)}{dz^{n}} \right|_{z=0} \quad \mbox{and} \quad P(n)
= \left.\frac{1}{n!} \frac{d^{n}H(z)}{dz^{n}} \right|_{z=0}.
\end{equation}
Compared to factorial moments which are also calculated from
probability generating functions, but at $z=1$, we can note a few
differences: \vspace{5mm}
\begin{quote}
\begin{itemize}
\item Calculated at $z=0$, more terms than for multiplicities.

\item Nested functions, lower order derivatives recurring.

\item Longer expressions, which on the other hand can be expressed
recursively
\end{itemize}
\end{quote}
\vspace{5mm}

The process of detection can be accounted for by the use of the
generating function $\varepsilon(z)$ of the binary probability
distribution of the number of neutrons detected per leaked
neutron:
\begin{equation}
\label{eq:ez} \varepsilon(z)=\epsilon z + (1-\epsilon).
\end{equation}
This model accounts for detection as one stochastic variable, and
is most easily thought of as a general detection efficiency for a
detector surrounding the sample, such as a multiplicity counter.
With some alterations one could use this model for finding
statistics for other detector setups as well.

Calculation of high order terms in the distribution requires high
order derivations of nested implicit functions.
\begin{quote}
\begin{itemize}
\item Symbolic derivations.

\item Using the symbolic language Mathematica.

\item Reevaluations fast due to symbolic expressions.
\end{itemize}
\end{quote}

%%% Introduction
\vspace{2cm}\begin{center}\pbox{0.8\columnwidth}{}{linewidth=2mm,framearc=0.1,linecolor=lightblue,fillstyle=gradient,gradangle=0,gradbegin=white,gradend=whiteblue,gradmidpoint=1.0,framesep=1em}{\begin{center}\Large
\bf Theoretical Treatment of Photons
\end{center}}\end{center}\vspace{1.25cm}


The generation of gamma photons is a more intricate process since
it is connected to the multiplication of neutrons and does not
have any self-multiplication. When incorporating absorption there
are a number of effects for the photons when looking at the main
parameter of the system which is the sample mass: \vspace{5mm}
\begin{quote}
\begin{itemize}
\item Increased mass means increased probability to induce
fission.

\item Absorption of neutrons lowers the number of induced
fissions.

\item Absorption of photons means that fewer photons escape the
sample and the number of visible photons decrease.
\end{itemize}
\end{quote}
\vspace{5mm} The master equations for gamma photons generated
within the sample read as follows:
\begin{equation}
\label{eq:gz} g(z)=(1-p)+ p \, r_f(z) \, q_f[g(z)],
\end{equation}
\begin{equation}
\label{eq:Gz} G(z)=r_s(z) \, q_s[g(z)].
\end{equation}
For describing the process of photon detection a few changes needs
to be made: \vspace{0.5cm}
\begin{quote}
\begin{itemize}
\item Need to account for absorption.

\item Need to account for detection.

\item The form of the equations need to be kept to find the
distributions easily.
\end{itemize}
\end{quote} \vspace{0.5cm}
All this is accomplished by the use of two extra equations:
\begin{equation}
\label{eq:lz} l(z)=l_{\gamma} z + (1-l_{\gamma}),
\end{equation}
\begin{equation}
\label{eq:egz} \varepsilon_{\gamma}(z)=\epsilon_{\gamma} z +
(1-\epsilon_{\gamma}).
\end{equation}
Here $l(z)$ comes from a master equation describing whether a
photon is absorbed or not with a special leakage probability
$l_{\gamma}$ which depends on the sample size, while
$\varepsilon_{\gamma}(z)$ describes the detection process with the
use of a detection efficiency $\epsilon_{\gamma}$ for detecting a
photon. Using these in the previous equations gives us the coupled
master equations used for finding the distribution of the
detections statistics:
\begin{equation}
\label{eq:gd} g_d(z) = g\big[l \{\varepsilon(z)\}\big] \quad ,
\quad G_d(z) =
 G\big[l \{\varepsilon(z)\}\big].
\end{equation}
The way the absorption and detection is put into the equations
makes the change in factorial moments very easy, we get the
leakage and detection probability raised to the same power as the
order of the moment enters the expression, and lower the numerical
values of the factorial moments.


%%% Section
\vspace{2cm}\begin{center}\pbox{0.8\columnwidth}{}{linewidth=2mm,framearc=0.1,linecolor=lightblue,fillstyle=gradient,gradangle=0,gradbegin=white,gradend=whiteblue,gradmidpoint=1.0,framesep=1em}{\begin{center}\Large
\bf Results \end{center}}\end{center}\vspace{1.25cm}

\vspace{-2.0cm}
%%% Figures:
\begin{center}
  % first argument: eps-file
  % second argument: stretching-factor relative to Column-width (<1)
  % optional argument: rotation angle (0-360), default=0
  \myfig{detneutBIG.eps}{0.7}
  \mycaption{The statistics for neutrons when a detection
efficiency of 50\% is incorporated into the model. The plot shows
that the statistics change and the likelihood of low detection
numbers are higher compared to how many bursts there are of that
multiplicity from the sample. This case is most representative for
a multiplicity counter that has a high total detection efficiency.
Samples contained 80 wt\% Pu-239 and 20 wt\% Pu-240.}
\end{center}

\vspace{-2.0cm}
\begin{center}
  % first argument: eps-file
  % second argument: stretching-factor relative to Column-width (<1)
  % optional argument: rotation angle (0-360), default=0
  \myfig{detgammaBIG.eps}{0.7}
  \mycaption{The statistics for photons when a detection
efficiency of 50\% is incorporated into the model (analytical),
compared to the MCNP-PoliMi results for generated photons.
Similarly to the inclusion of absorption the inclusion of the
detection process lowers the probabilities of observing high
multiplicities.}
\end{center}

The analytical expressions derived for the number distribution
were evaluated using the nuclear data taken from the Monte Carlo
code MCNP-PoliMi. For neutrons, inclusion of the absorption does
not affect the probabilities in a significant way compared to the
non-absorbing case. This is to be expected for the samples
investigated in this work since they are comprised of high-$Z$
material. The three samples investigated were of different masses
with a mix of 80 wt\% Pu-239 and 20 wt\% Pu-240. A much larger
effect is seen when incorporating a detection efficiency. If it is
low, then probabilities to see several neutrons decrease
drastically even when considering high multiplicities in big
samples.


%%% Section
%\vspace{2cm}\begin{center}\pbox{0.8\columnwidth}{}{linewidth=2mm,framearc=0.1,linecolor=lightblue,fillstyle=gradient,gradangle=0,gradbegin=white,gradend=whiteblue,gradmidpoint=1.0,framesep=1em}{\begin{center}\Large
%\bf Numerical Treatment - Photons\end{center}}\end{center}
%\vspace{1.25cm}

\vspace{-1.5cm}
%%% Figures:
\begin{center}
  % first argument: eps-file
  % second argument: stretching-factor relative to Column-width (<1)
  % optional argument: rotation angle (0-360), default=0
  \myfig{MCgraph.eps}{0.9}
  \mycaption{The detection statistics calculated for a setup of
  six scintillator detectors sensitive to both photons and neutrons,
  with detection efficiencies of 0.00546 and 0.00549 for neutrons
  and photons respectively. MCNP-PoliMi was used to generate these
  results for a 9.00 kg sample of 80 wt\% Pu-239 and 20 wt\% Pu-240.}
\end{center}
\vspace{-0.5cm}

For photons the effect of absorption is very big and one can see a
massive effect of the self-shielding in all samples. Looking at
what happens in different sample masses the following can be
observed:
%\vspace{5mm}
\begin{quote}
\begin{itemize}
    \item Larger mass means higher numbers of neutrons and photons are
    generated.

    \item Heavier mass means more shielding for the photons.

    \item The combined effect is dominated by the shielding.

    \item Smaller samples have larger numbers of photons escaping
compared to larger samples per initial source event.
\end{itemize}
\end{quote}
%\vspace{5mm}
The effect of detection is straight-forward and seen as an
attenuation of the probabilities in much the same way as the
absorption, with the change that the absorption is different for
different sample masses, while the detection has the same effect
for all masses, since the detection efficiency is the same.

%%% Section
\vspace{2cm}\begin{center}\pbox{0.8\columnwidth}{}{linewidth=2mm,framearc=0.1,linecolor=lightblue,fillstyle=gradient,gradangle=0,gradbegin=white,gradend=whiteblue,gradmidpoint=1.0,framesep=1em}{\begin{center}\Large
\bf Conclusions\end{center}}\end{center}\vspace{1.25cm}


We have used the symbolic computation code Mathematica to
calculate high order terms of the number distribution of neutrons
and photons from fissile samples with the inclusion of absorption.
The results show that when absorption is accounted for, the number
of photons emerging from the sample will decrease significantly,
whereas the neutrons are not affected to the same extent. The
multiplicities of photons leaving the sample could decrease so
much that neutron multiplicities become higher. With the
introduction of the detection process into the model of the leaked
neutrons and photons, it is possible to simulate the detector
response and find the probabilities for different multiplicities
of both neutrons and photons. Using this information, one could
assess different sample masses with regards to what type of
emission has the higher multiplicity when using detectors for
non-destructive assay of the material.

% By introducing absorption into the models describing the
%number distribution of neutrons and gamma photons emitted from a
%fissile sample, one can obtain the actual number of escaped
%neutrons and photons which are available for detection. The
%results show that the absorption in a plutonium sample will have a
%much larger effect on photons as compared to neutrons. This means
%that there is a mass limit, above which neutrons become a better
%tool for NDA-studies for heavy samples. The change for factorial
%moments is very straightforward in the case of photons where a
%leakage probability, raised to the same power as the order of the
%factorial moment, decreases the numerical values of the factorial
%moments.



%%% References
%\bibliographystyle{alpha}
%\bibliography{poster.bib}


\end{multicols}

\end{poster}

\end{document}
