\documentclass[12pt,a4paper]{article}

% if you use PostScript figures in your article
% use the graphics package for simple commands
% \usepackage{graphics}
% or use the graphicx package for more complicated commands
% \usepackage{graphicx}
% or use the epsfig package if you prefer to use the old commands
% \usepackage{epsfig}

% The amssymb package provides various useful mathematical symbols
\usepackage{amssymb}
\usepackage{amsmath}
\usepackage[dvips]{graphicx}
\usepackage{here}

\usepackage[latin1]{inputenc}    % Accept european-encoded (latin1) characters.

%\usepackage{pspalatino}
%\usepackage{palatino}
\usepackage{times}
%\usepackage[figuresright]{rotating}

\setlength{\textwidth}{430pt} \setlength{\textheight}{630pt}
\setlength{\hoffset}{-20pt} \setlength{\voffset}{-40pt}

\begin{document}



\begin{center}

  \textbf{\Large{The number distribution of neutrons and gamma photons generated in a multiplying sample}}

 \vspace{0.4 cm}

  \large{Andreas Enqvist\footnote{Corresponding author.}$^{a}$, Imre P\'{a}zsit$^{a}$, Sara Pozzi$^{b}$}
\vspace{0.2 cm}

  $^{a}$Department of Nuclear Engineering, Chalmers University of Technology, SE-412 96 G\"{o}teborg,
  Sweden\\
  $^{b}$Oak Ridge National Laboratory, PO Box 2008 Ms6010, Oak Ridge TN 37831-6010, USA

  \vspace{1 cm} % Vertical space

  %\includegraphics[width=10cm]{nice.eps}

  \ % Empty paragraph

\end{center}

\begin{abstract}

The subject of this paper is an analytical derivation of the full
probability distribution of the number of neutrons and photons
generated in a sample with internal multiplication by one source

\end{abstract}

\section{Introduction}

\end{document}
