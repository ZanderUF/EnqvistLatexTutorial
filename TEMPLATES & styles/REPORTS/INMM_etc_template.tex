\documentclass[11pt,twoside]{article}  %might need to use this
%             version on computers with other TeX-distributions,
%             frontpage might look odd, but text should work.
\usepackage[T1]{fontenc}

\usepackage{tabls}
\usepackage{afterpage}
\usepackage{amsmath}
\usepackage[dvips]{graphicx}
\usepackage{subfigure}
\usepackage{here}
\usepackage{color}
\usepackage{graphicx}
\usepackage{lastpage} % total page count
%\usepackage{cites}
%\usepackage{epsf}

%\usepackage{pspalatino}
%\usepackage{palatino}
%\usepackage{times}
\usepackage{pslatex}
%\usepackage{mathtime}
%\usepackage[figuresright]{rotating}

\setlength{\textwidth}{6.5in}
\setlength{\textheight}{9.0in}
\setlength{\marginparsep}{0pt}
\setlength{\marginparwidth}{0pt}
\setlength{\oddsidemargin}{0pt}
\setlength{\evensidemargin}{0pt}
\setlength{\hoffset}{0pt}
\setlength{\topmargin}{16pt}
\setlength{\headsep}{16.8pt}
\setlength{\headheight}{14pt}
\setlength{\footskip}{0.4in}



% Nicer headers and footers
%
% \fancyhead[L/C/R]{} to change headers
% \fancyfoot[L/C/R]{} to change footers
\usepackage{fancyhdr}
\renewcommand{\headrulewidth}{0pt}
\newcommand{\authorHead}      % Author's names here
   {Andreas Enqvist, Marek Flaska, Sara A. Pozzi, Jennifer L. Dolan and David L. Chichester}
\newcommand{\shortTitle}      % Short title here
   {Evaluation of a Combined Neutron and Gamma-Ray Multiplicity Counting System}
\fancyhf{}

\pagestyle{fancy}

%%%%%%%%%%%%%%%%%%%%%%%%%%%%%%%%%%%%%%%%%%%%%%%%%%%%%%%%%%%%%%%%%%%%%
%
%   BEGIN DOCUMENT
%
%%%%%%%%%%%%%%%%%%%%%%%%%%%%%%%%%%%%%%%%%%%%%%%%%%%%%%%%%%%%%%%%%%%%%


\begin{document}

\thispagestyle{empty}  %added due to header changes (no header on title page, looks better)

\afterpage{               %Remove/comment out the whole section if you have no desire to use headers/footers
\fancyhead[CO]{   %Centered header on ODD! pages
{\scriptsize \shortTitle}}
\fancyhead[CE]{    %Centered header on EVEN! pages, to make the same use:  "\fancyhead[CE,CO]{..." or just "\fancyhead[C]{...."
{\scriptsize \authorHead}}
\fancyfoot[CO,CE] {\thepage}                        %page number  "X"
%\fancyfoot[CO,CE] {\thepage/\pageref{LastPage}}    % page number as "X/YY"
\pagestyle{fancy}
}

\begin{center}


  \textbf{\Large{Evaluation of a Combined Neutron and Gamma-Ray Multiplicity  \\ \vspace{2mm}Counting System }} % \\ \vspace{2mm} }}

 \vspace{0.4 cm}

  Andreas Enqvist\footnote{Corresponding author (\texttt{enqvist@umich.edu}).}$^{a}$, Marek Flaska$^{a}$, Sara A. Pozzi$^{a}$, Jennifer L. Dolan$^{a}$ and David L. Chichester$^{b}$
\vspace{0.2 cm}

  $^{a}$Department of Nuclear Engineering and Radiological Sciences, University of Michigan, Ann Arbor, MI 48109, USA\\
  $^{b}$Idaho National Laboratory, Idaho Falls, ID 83415

  \vspace{0.3 cm} % Vertical space

  %\includegraphics[width=10cm]{nice.eps}

  \ % Empty paragraph

\end{center}



\begin{abstract}

Multiplicity counters for neutron assay have been extensively used
in materials control and accountability for nonproliferation and
nuclear safeguards. Typically, neutron coincidence counters are
utilized in these fields. In this paper we present a measurement
system that makes use not only of neutron (n) multiplicity
counting but also of gamma ray ($\gamma$) multiplicity counting
and the combined higher-order multiplets containing both neutrons
and gamma rays. The benefit of this approach is in using both
particle types available from the sample, leading to a reduction
in measurement time when using more measurables. We present
measurement results of n, $\gamma$, nn, n$\gamma$, $\gamma\gamma$,
nnn, nn$\gamma$, n$\gamma\gamma$ and $\gamma\gamma\gamma$
multiplets emitted by a $^{252}$Cf source. Additionally, the system was also tested using a much more complex source in the form of MOX fuel pins. Such a complex source can more easily reveal short comings and complications of the system. The dual radiation measuring system proposed here can use
extra measurables either to improve the statistics when compared
to a neutron-only system, or alternatively be used for extended
analysis and interpretation of sample parameters.

The study presented here provides an answer to the questions
regarding the ability to measure both neutron and gamma ray
coincidences at once. The results show that the measurements
system proposed in this study has a potential to be valuable in
the area of nuclear nonproliferation and homeland security.

\end{abstract}

\emph{Key Words}: nuclear nonproliferation, pulse shape discrimination,
liquid scintillation detectors, multiplicity counting, coincidences.


% main text
\section{Introduction}
\label{Introduction}

Neutron multiplicity counters have been used for several decades,
mostly as thermal multiplicity counters based on $^{3}$He
detectors \cite{ensslin98} or as counters with slightly wider
neutron energy ranges \cite{ENMC}. These systems have very good
detection efficiency; however, the time resolution suffers from the
required moderation of the neutrons. A number of new measurement
options have recently become available with the advancements in
applying organic liquid scintillation detectors within the areas
of nuclear nonproliferation and materials control and
accountability and with the developments in fast digitizer
technology.

Organic scintillation detectors can provide some energy
spectroscopy \cite{Klein} as well as excellent die-away
properties, thus reducing the number of accidentals in
measurements. Furthermore, the pulse shape can be used to
discriminate between different types of incoming radiation. For
example, neutron pulses can be separated out from the gamma ray
pulses. This means that data analysis techniques such as neutron
spectrum unfolding from the measured pulse height distributions
(PHDs) \cite{Xu,Reginatto} can be applied, even in the presence of
gamma ray radiation.

Recently, liquid scintillation detectors have been investigated
for use in neutron multiplicity counters \cite{LSMC}. The main
benefit of such a design is that it is much less sensitive to
accidental coincidences than the traditional approach based on
$^{3}$He detectors due to much faster time response. Thus, the liquid scintillators are suitable
for use even on samples with high single-neutron background, such
as ($\alpha$, n) rates exhibited in some samples containing different
isotopes of plutonium. However, Frame et al. \cite{LSMC} estimated
that such a system will have problems with neutron classification
at energies below 2 MeV, as well as having a low total detection
efficiency. Such problems might be corrected with alternative
designs of the detector. It can be noted that all gamma ray
pulses would be ignored in this type of design.

Fission multiplicity counters \cite{Sher} use systems of plastic
scintillation detectors sensitive to both fast neutrons and gamma
rays. However, these counters offer no possibility of discriminating the
gamma rays from the neutrons, thus only the total fission
multiplicity is measured, rather than a neutron or gamma ray
multiplicity. More recent developments, especially those using the
Nuclear Materials Identification System \cite{NMIS}, allowed
temporal discrimination of gamma rays from neutrons to further
enhance the analysis of nuclear materials.

%The gamma rays were not treated explicitly in the analysis that
%relate the combined multiplicity to effective mass,
%($\alpha,n$)-rate, multiplication and detection efficiency, so
%only when using gamma ray shielding were the systems allowing
%accurate multiplicity analysis.

We suggest a novel approach for finding sample characteristics
based on the use of multiple liquid scintillators, a fast
digitizing system, and advanced analysis algorithms. The proposed
system uses pulse shape discrimination (PSD) \cite{Flaska} to
determine what particle caused the detection pulse. The excellent
timing properties of liquid scintillators are used to find out
what pulses are correlated in time and the particle types that
created them. This approach has the benefit of providing more
measurables compared to the case of pure neutron multiplicity
counting, while having a few extra parameters - but not more than
the amount of added measurables - which could be used in a number
of ways to determine the sample characteristics
\cite{PazPal08,AEP,EPA}.

This paper describes the measurement system used for performing
multiplicity measurements using neutron and gamma ray sources such
as $^{252}$Cf and mixed-oxide fuel. The approach builds upon
previously described cross-correlation measurements
\cite{CC,Ispra}. The PSD will be described in detail as one of the
key parameters for the investigation of the proposed approach. The
results of initial assessment will then be discussed and future
possible improvements along with potential drawbacks of the
proposed system.


\section{Description of Experimental Setup}
\label{sec:Setup}

The measurements were performed at the University of Michigan,
Department of Nuclear Engineering and Radiological Sciences -
where the measurement system was developed - and at Idaho National Laboratory (INL) during the summer of 2009. The
measurement system consists of four EJ-309 cylindrical
scintillation detectors with a height of 13.3~cm and a diameter of
13~cm. This liquid has a high flash point (144 $^\circ$C), and low
chemical toxicity.

In the measurements (see the measurement setup in Fig. \ref{fig:setup}), the detectors were
placed at 90 degrees with the source-detector distance kept
constant. The data acquisition is performed with a 250-MHz,
12-bit, 8-channel CAEN V1720 digitizer. Four channels were used,
each being able to trigger independently on the incoming detector
pulses. The data from all channels were stored and then
subsequently analyzed to determine if any additional detectors, in
addition to the trigger detector, contained time-correlated pulses.
The detectors were calibrated using a $^{\textrm{137}}$Cs source.

\begin{figure}[!htb]
\centering
\includegraphics[width=9cm]{SNC00110(2).eps}
\caption{Measurement setup.}
\label{fig:setup}
\end{figure}

A threshold of approximately 70 keVee corresponding to a neutron
energy deposited of approximately 450 keV, was applied to achieve
excellent PSD performance. The 70-keVee threshold was chosen based
on previous investigations focused on the accuracy of particle
discrimination. At this threshold, a gamma rejection of
approximately 1/1000 is expected. The $^{252}$Cf source was placed
at different distances to determine the effect of
changing absolute detector efficiency, while the MOX measurement presented here were performed for a single larger distance with 5~cm lead shielding to
suppress high gamma ray production. The count rates when using all
four detectors varied from approximately 4.5~kHz to 15~kHz, with
the background of approximately 1~kHz. The measurement setups
were as follows:
\begin{itemize}
\item $^{252}$Cf, 15~cm source--detector distance,

\item $^{252}$Cf, 20~cm source--detector distance,

\item $^{252}$Cf, 30~cm source--detector distance,

\item 100 MOX fuel pins shielded with 5~cm of lead, 40~cm
    source--detector distance.
\end{itemize}
The measurement times varied between 10 and 30 minutes resulting
in approximately 10 million recorded events per setup.


\section{Pulse Shape Discrimination and Pulse Timing}
\label{sec:PSD}

PSD and pulse timing are performed with an offline algorithm.
Traditionally, PSD can be performed based on charge integration or
differentiation \cite{Adams,Sperr}. In previous work, we have used
charge integration and analyzed the tail-to-total integral ratio
versus pulse height \cite{CC}. In this work, the total integral is
calculated from the start of the pulse to 220~ns past the pulse
maximum, while the tail integral is calculated from 20~ns to
220~ns past the pulse maximum. These values were previously
determined by analyzing the figure of merit for the PSD as a
function of the time ranges for the pulse integration.

%\begin{figure}%[htb]
%\begin{center}
%  % Requires \usepackage{graphicx}
% \subfigure[Full range.]{\label{fig:PSDf}
%\includegraphics[width=7cm]{PSDfull2.eps}}\vspace{-2mm} \\
% \subfigure[Zoomed in.]{\label{fig:PSDz}
%\includegraphics[width=7cm]{PSDzoom.eps}}\vspace{-2mm}
%%  \includegraphics[width=7.5cm]{2colltot1a.eps} \\
%%  \includegraphics[width=7.5cm]{figures/2colltot10b.eps}\\ %\vspace{-9mm}
%\caption{Tail vs. total integrals for a $^{252}$Cf measurement.
%The data shown are for a single EJ-309 detector, and the discrimination line
%indicates which pulses are created by neutrons (above the line) and by gamma rays (below the line).}
%\label{fig:PSD}
%\end{center}
%\end{figure}

%A linear discrimination boundary is used to discriminate neutrons
%from gamma rays (Fig. \ref{fig:PSD}). This method show high
%accuracy, but can unfortunately not be quantified with simple
%figures of merit (FOM) due to the nonlinear discrimination curve
%chosen. The FOM is defined as follows: FOM = distance between
%peaks of neutron and gamma distributions / (FWHM$_{\gamma}$ +
%FWHM$_{neutron}$)

Pulse timing is performed on the digitized pulse data by
simulating a constant fraction discriminator where the the start
of the pulse is defined as a fixed percent of the peak value. The timing is
then less sensitive to the pulse amplitude. Using interpolation
between the data values allows for increased time resolution from
the initial 4~ns to approximately 1~ns. The sharp rise of the
pulse edge means linear interpolation can be used without large
losses of accuracy. The gamma ray--gamma ray peak in the
cross-correlation measurements was used to test the timing
accuracy and a FWHM of less than 1.5~ns was observed.

\begin{figure}[!htb]
\centering
\includegraphics[width=10.5cm]{pulsesamples.eps}
\caption{Sample pulses from the data-acquisition system.}
\label{fig:pulses}
\end{figure}


\section{Neutron and Gamma Ray Multiplicities} \label{sec:ngmult}

%In a $^{3}$He based neutron multiplicity counter, there is no
%concept of energy threshold since the neutrons are not detected at
%their initial energy but instead needs to be moderated to lower
%energies first. The proposed system however which depends on
%scattering in the scintillation detectors have a certain
%threshold, and for this type of system to be utilizable
%measurements were done to find all 9 multiples up to third order,
%and verifying that it can be done successfully for a fissile
%sample.

The benefit of the proposed system is its speed and robustness
which is due to the fact that several quantities are measured
(nine multiplets up to the third order). Further, the system
operates on short time scales (300~ns is the width of a typical
pulse) due to the removal of neutron moderation when compared to a
$^{3}$He neutron multiplicity counter. While a $^{3}$He counter
design usually encompasses the entire sample to maximize the
neutron moderation and consequently the detection efficiency, a
scintillation system could be made with a sparse layout and thus
be a candidate for a portable system. Such a design requires that
detection rates of multiplets are sufficiently high even with
relatively low absolute detector efficiency. For allowing the
unfolding of sample parameters such as proposed in \cite{AEP},
successful measurement systems for all the multiplets need to be
devised.


\subsection{Measurement Results and Discussion} \label{sec:Meas}

The measurements were performed with a $^{252}$Cf source producing
19500 fissions/second \cite{Valentine99} and a shielded MOX container. All pulses that could constitute a real
coincidence, considering time of flight for the neutrons and the
total size of the system (up to approximately 50-ns difference),
are fully registered with the digitizer in a window of 400~ns (100
sampling points).

Table \ref{tab:Cf} shows the measured rates of multiplets for
various source--detector distances from the $^{252}$Cf source. The
results indicate that for measurement times in the order of a few
minutes good accuracy for the singlet and doublet rates can be
achieved.

\subsection{Measurements with $^{252}$Cf} \label{sec:MeasCf}

%\begin{table}
%  \centering
%\begin{tabular}{|l|r|r|r|}
%  \hline
%  % after \\: \hline or \cline{col1-col2} \cline{col3-col4} ...
%\multicolumn{4}{|c|}{Count rates:}    \\     \hline
%& 30cm & 20cm & 15cm    \\     \hline
%$R_{n}$  & 690.4 & 1338.6 & 1947.4    \\
%$R_{p}$  & 3119.9 & 5335.6 & 7256.6    \\
%$R_{nn}$  &  13.26 &   51.46 &  114.3    \\
%$R_{np}$  &  52.31 &  197.3 &  409.3    \\
%$R_{pp}$  &  63.36 &  227.7 &  455.0    \\
%$R_{nnn}$  &  0.101 &   0.802 &   2.959    \\
%$R_{nnp}$  &  0.645 &   5.503 &   18.72    \\
%$R_{npp}$  &  1.540 &   12.01 &   38.72    \\
%$R_{ppp}$  &  1.179 &   8.803 &   27.04 \\ \hline
%\multicolumn{4}{|c|}{Measurement time (s):}\\  \hline
% & 1735.8 & 1475.3 & 1054.6  \\  \hline
%\end{tabular}
%\caption{$^{252}$Cf count rates for different source--detector
%distances.}\label{tab:Cf}
%\end{table}

\begin{figure}[!h]
\centering
\includegraphics[width=10.5cm]{rates3.eps}
\caption{Detection rates of different multiplets at different
distances for the $^{252}$Cf source.}
\label{fig:rates}
\end{figure}

\begin{table}
  \centering
\begin{tabular}{|l|r|r||r|r||r|r|}
  \hline
  % after \\: \hline or \cline{col1-col2} \cline{col3-col4} ...
\multicolumn{7}{|c|}{Count rates (s$^{-1}$) and statistical errors:}    \\     \hline
& 30cm & $\pm 1 \sigma$ & 20cm & $\pm 1 \sigma$  & 15cm & $\pm 1 \sigma$   \\     \hline
$R_{n}$  & 690.4 & 0.63 & 1338.6 & 0.95 & 1947.4  & 1.36    \\
$R_{\gamma}$  & 3119.9 & 1.34 & 5335.6 & 1.90 & 7256.6 & 2.62    \\
$R_{nn}$  &  13.3 & 0.09 & 51.5 & 0.19 &  114.3 & 0.33    \\
$R_{n\gamma}$  &  52.3 & 0.17 & 197.3 & 0.37 &  409.3 & 0.62    \\
$R_{\gamma\gamma}$  &  63.4 & 0.19 & 227.7 & 0.39 &  455.0 & 0.66    \\
$R_{nnn}$  &  0.1 & 0.01 &   0.8 & 0.02 &   3.0   & 0.05   \\
$R_{nn\gamma}$  &  0.6 & 0.02 &   5.5 & 0.06 &   18.7   & 0.13  \\
$R_{n\gamma\gamma}$  &  1.5 & 0.03 &   12.0 & 0.09 &   38.7   & 0.19   \\
$R_{\gamma\gamma\gamma}$  &  1.2 & 0.03 &   8.8 & 0.08 &   27.0   & 0.16 \\ \hline
& \multicolumn{6}{|c|}{Measurement time (s):}\\  \cline{1-7}
\,\,\, $T$ & \multicolumn{2}{|c||}{1735.8} & \multicolumn{2}{|c||}{1475.3} & \multicolumn{2}{|c|}{1054.6} \\ \cline{1-7}
\end{tabular}
\caption{$^{252}$Cf count rates for different source--detector
distances and all possible neutron, gamma ray and mixed multiplets
up to the third order. The statistical error is calculated as the
square root of the total counts divided by the total measurement
time.}\label{tab:Cf}
\end{table}


\begin{table}
  \centering
\begin{tabular}{|l|r|r|r|}
  \hline
%  % after \\: \hline or \cline{col1-col2} \cline{col3-col4} ...
\multicolumn{4}{|c|}{Ratios:}   \\     \hline
 & 30cm & 20cm & 15cm\\     \hline
$R_{n}/R_{nn}$  & 52.1 & 26.0 & 17.0      \\
$R_{n}/R_{nnn}$   & 6809.3 & 1669.3 & 658.2   \\
$R_{nn}/R_{nnn}$  & 130.7 & 64.2 & 38.6   \\
$R_{\gamma}/R_{\gamma\gamma}$   & 49.2 & 23.4 & 15.9   \\
$R_{\gamma}/R_{\gamma\gamma\gamma}$   & 2646.8 & 606.1 & 268.4  \\
$R_{\gamma\gamma}/R_{\gamma\gamma\gamma}$  & 53.7 & 25.9 & 16.8 \\
$R_{n}/R_{\gamma}$   & 0.22 &  0.25 & 0.27 \\
$R_{nn}/R_{\gamma\gamma}$  & 0.21 & 0.23 & 0.25  \\
$R_{nnn}/R_{\gamma\gamma\gamma}$  & 0.09 & 0.09 & 0.11 \\ \hline
\end{tabular}
\caption{Different $^{252}$Cf ratios of the count rates could be
used to obtain geometrical information.}\label{tab:ratio}
\end{table}

%The analysis scripts used to analyze the data and classify the
%multiples have computing times on the order of one to a few
%minutes which is less than the actual measurement times. It would
%be beneficial to run them simultaneously with the data
%acquisition, however that is not currently possible due to the
%extra strain that it puts on the computer system. This is one of
%the possible improvements that could be addressed in future work.



The rates of the different multiplets for the $^{252}$Cf source
are shown in Fig. \ref{fig:rates}. It is evident that the
source--detector distance has a large impact on the higher-order
multiplets, which increase by more than an order of magnitude when
the distance is decreased from 30~cm to 15~cm. With reduced
distance, the rate of the higher-order multiplets increases more
than the total count rate at the largest distance. Optimal rates
are considered those where the measurement uncertainty for the
high-order multiplets is low (acceptable) for reasonably short
measurement times. With detection rates for gamma rays being
higher, this might indicate that the minimum detectable activity
could be reduced when also accounting for gamma multiplets
compared to pure neutron multiplicity.

Table \ref{tab:ratio} reports a number of the ratios of rates that
could be of interest, such as ratios between different multiplets
of neutrons, which show the strong dependence on absolute detector
efficiency with varying distance. More intricate ratios such as
those between the same multiplet of neutrons and gamma rays could
be used for possible PSD error analysis. It has been noticed in the measurements that the neutron--neutron (nn) doublets are
clearly depending on the orientation of the detectors, with
opposing detectors recording more than 50\% more nn doublets compared
to adjacent detectors. This corresponds well with the findings in
Ref. \cite{Aniso}.

The values of the ratios $R_{n}/R_{\gamma}$,
$R_{nn}/R_{\gamma\gamma}$, $R_{nnn}/R_{\gamma\gamma\gamma}$ should
be also emphasized, because they are almost independent of
the source--detector distance. However, they do change depending
on if they are the singlet, doublet, or triplet neutron to gamma
ray ratios, since the number of gamma rays and neutrons produced
per single fission significantly varies. Specifically, up to 20
gamma rays, 10 neutrons can be emitted from a single fission,
which can significantly change the above mentioned ratios. With
the spontaneous-fission distribution of different isotopes being
different, this set of ratio values could be used for isotopic
determination. The singlet ratios also depend on the existence of
($\alpha$, n) events, which needs to be taken into account for the
potential identification of the isotopes present in the measured
sample.

One particular character of the multiplets is the behaviour of
their magnitude when neutrons are successively replaced by gamma
rays in the same order of multiplet. This can be easily seen in
Table \ref{tab:Cf} and especially in Fig. \ref{fig:rates}, since
the values are arranged just in this order. For the doublets,
starting with the pure neutron doublet, replacing a neutron with a
gamma ray always leads to a higher value, although the increase is
smaller in the second step than in the first. For the triplets,
the change of the magnitude is not monotonic; the magnitude of the
pure gamma triplet rate is smaller than that of the  n$\gamma
\gamma$ triplet rate. This shows the special value of the use of
the mixed rates, as it was also mentioned in \cite{AEP,EPA}.

\subsection{Measurements with MOX} \label{sec:MOX}

The $^{252}$Cf source was useful for the initial test for the system, however if one wishes to use the system for assaying of ``real samples'' much more complicated compositions and radiological signatures might be encountered. To test how the data analysis copes with such a scenario, measurements of MOX samples performed at INL are used for the system testing.

\begin{figure}[!htb]
\centering
\includegraphics[width=10cm]{fuelpins.eps}
\caption{Canister of MOX fuel pins used in measurements taken at INL.}
\label{fig:setup3}
\end{figure}

The MOX samples have a very different geometry compared to what a point-like $^{252}$Cf source has. Figure \ref{fig:setup3} shows the can in which up to 90 short fuel pins were contained. The sample was then placed in the center between four EJ-309 detectors located at right angles and with the same distance (40 cm) to the source (cf. Fig. \ref{fig:setup}). Due to the strong intrinsic source of gamma rays, the detectors were shielded with 5 cm of lead. This mimics detection of gamma rays present in shielded sources. Of the first 9 multiplicities 6 of them contain gamma rays, meaning that most of the detection rates will heavily suppressed.

\begin{table}
\centering
\begin{tabular}{|l|r|r|}
  \hline
  % after \\: \hline or \cline{col1-col2} \cline{col3-col4} ...
\multicolumn{3}{|c|}{MOX (sample 129):} \\    \hline
& Rate & $\pm 1 \sigma$ \\    \hline
  $R_{n}$    & 4822.2 &    2.6  \\
  $R_{\gamma}$  & 10027.1 &    3.7  \\
  $R_{nn}$    &  21.92 &   0.17  \\
  $R_{n\gamma}$   &  28.31 &   0.20  \\
  $R_{\gamma\gamma}$  &  28.40 &   0.20  \\
  $R_{nnn}$    &  0.094 &  0.011  \\
  $R_{nn\gamma}$  &  0.105 &  0.012  \\
  $R_{n\gamma\gamma}$  &  0.056 &  0.009  \\
  $R_{\gamma\gamma\gamma}$ &  0.040 &  0.008  \\
   \hline
\end{tabular}
\caption{Measured multiplet rates for 90 MOX pins at 40 cm distance. The total
measurement time was 716 seconds.}\label{tab:MOX}
\end{table}

\begin{figure}[!htb]
\centering
\includegraphics[width=10.5cm]{40cmMOX.eps}
\caption{Detection rates for different multiplets for the MOX fuel pins source. Error bars ($\pm 1 \sigma$) are included but are very small except for the case of triplet rates.}
\label{fig:ratesMOX}
\end{figure}

The measurement time was to 12 minutes; however, during that time a total of more than 10 millions events were recorded, meaning that the statistics of the singlet and doublet rates are good also after shorter measurements. In the case of the more rare triplet multiplets, there is the issue shielding in addition to larger source detector distance which also reduces the geometrical detector efficiency of the setup when compared to the previous californium measurements.

It is obvious that rates can be also successfully acquired for the MOX sample. To reach the same low level of statistical uncertainty on the triplet rates as for the $^{252}$Cf would require a longer measurement time (also when normalizing to the same distance and relative source strength). As a rough indicator of the presence of lead shielding it can be noted that the ratio between neutron and gamma ray detection rates decreases for higher orders, which is the reverse behavior expected from the higher initial multiplicity of gamma rays from fission events: $R_{n}/R_{\gamma}= 0.48, R_{nn}/R_{\gamma\gamma}= 0.77, R_{nnn}/R_{\gamma\gamma\gamma}= 2.31$.

%\subsection{Measurements with Pu-Be} \label{sec:MeasPuBe}
%
%Measurements were performed with a strong Pu-Be source, which
%emits one neutron per ($\alpha$, n) reaction accompanied with one
%or more gamma rays. The source also emits a large number of
%uncorrelated gamma rays, which was the reason for placing 5 cm of
%lead shielding around the source. Measurement results are given in
%Table \ref{tab:PuBe}: as can be noted, very few multiplet above
%the first order are detected. Even n$\gamma$ doublets which should
%have high frequency in connection to the ($\alpha$, n) events, are
%severely suppressed by the Pb shielding.
%
%Table \ref{tab:PuBe} shows that the singlets of neutrons and gamma
%rays are very dominating, while triplets are essentially
%non-existent due to the low number of correlated events. As
%expected, measured singlet-to-doublet and singlet-to-triplet
%ratios are much lower for the Pu-Be source than for the $^{252}$Cf
%source, which is caused by lack of time-correlated detection
%events. These quantities could be used as a quick estimate of
%sample types, such as fissile versus non-fissile. In the case of
%the fissile $^{252}$Cf source, the triplet rates are approximately
%a factor 1000 lower than the singlet rates (see $R_{n}/R_{nnn}$
%and $R_{\gamma}/R_{\gamma\gamma\gamma}$ in Table \ref{tab:ratio}),
%while for the Pu-Be source, high-order correlated events are
%almost nonexistent, and occur a factor 1,000,000 times more rarely
%than a single event. In addition, other source properties such as
%ratios of gamma rays to neutrons for different radiation shields
%that would indicate the source type could be obtained. The
%multiplicity system discussed here could be `calibrated' to allow
%identification of various neutron and gamma ray sources and
%nuclear materials. The additional benefit of measuring also gamma
%rays could depend on the assayed material, making the approach
%more suitable to some types of measurements.
%
%
%\begin{table}
%\centering
%\begin{tabular}{|l|r|r|}
%  \hline
%  % after \\: \hline or \cline{col1-col2} \cline{col3-col4} ...
%\multicolumn{3}{|c|}{PuBu source:} \\    \hline
%& Rate & $\pm 1 \sigma$ \\    \hline
%  $R_{n}$    &     8160.3 & 3.2494 \\
%  $R_{\gamma}$    &     3283.3 & 2.0611 \\
%  $R_{nn}$   &     7.0647 & 0.0956 \\
%  $R_{n\gamma}$   &     10.571 & 0.1170 \\
%  $R_{\gamma\gamma}$   &     3.9179 & 0.0712 \\
%  $R_{nnn}$   &     0.0013 & 0.0013 \\
%  $R_{nn\gamma}$   &     0.0116 & 0.0039 \\
%  $R_{n\gamma\gamma}$   &     0.0078 & 0.0032 \\
%  $R_{\gamma\gamma\gamma}$   &     0.0103 & 0.0037 \\
%  \hline
%\end{tabular}
%\caption{Measured multiplet rates for the Pu-Be source. The
%measurement time was 772 seconds.}\label{tab:PuBe}
%\end{table}
%
%\begin{figure}[!htb]
%\centering
%\includegraphics[width=8cm]{ratesPuBe2.eps}
%\caption{Detection rates for different multiplets for the Pu-Be source.}
%\label{fig:ratesPuBe}
%\end{figure}




\subsection{Analysis of PSD error}

It is clear that the PSD performance is vital for the accuracy of
the measurement method. Especially when looking at the higher
order multiplets, the chance of it being wrongly classified
increases with the amount of detector pulses that make up each
specific multiple. There are some promising features of the system
that could be used to both enhance the understanding of the PSD
performance as well as create correction factors for improved
assay which would not be available in previous liquid
scintillation designs \cite{LSMC}. The effect of misclassification
on the measured rates depends on the ratios of different
multiplets, which in turn depends on detection efficiency, sample
multiplicity distribution and shielding.

Assume that the count rates are related as $R_{nnn} =
0.1R_{nn\gamma} = 0.1R_{n\gamma\gamma} =
0.1R_{\gamma\gamma\gamma}$, which is only a rough estimate of what
was observed in the $^{252}$Cf measurements. It means that if 1\% of all pulses
are misclassified the value of $R_{\gamma\gamma\gamma}$ will be
underestimated by 2\%, while the $R_{nnn}$ value will be
overestimated by 7\%. Similarly, if the PSD error is 5\% then the
value of $R_{\gamma\gamma\gamma}$ could be off by 9.5\%, while the
$R_{nnn}$ value could be off by 33\%. The values are related as
follows:
\begin{equation}\label{eq:Rnnn}
R_{nnn} =  R'_{nnn}(1-\epsilon)^{3} + R'_{nn\gamma}\epsilon(1-\epsilon)^{2}
+ R'_{n\gamma\gamma}\epsilon^{2}(1-\epsilon)+ R'_{\gamma\gamma\gamma}\epsilon^{3},
\end{equation}
\begin{equation}\label{eq:Rppp}
R_{\gamma\gamma\gamma} = R'_{\gamma\gamma\gamma}(1-\epsilon)^{3} +
R'_{n\gamma\gamma}\epsilon(1-\epsilon)^{2} + R'_{nn\gamma}\epsilon^{2}(1-\epsilon)
+ R'_{nnn}\epsilon^{3},
\end{equation}
where $\epsilon$ is the PSD error and $R'_{xxx}$ are the real
rates and $R_{xxx}$ are the measured rates.

When considering a pure neutron multiplicity assay, the resulting
errors on the measured rates would be high if the PSD error was
very high, but with the additional information available, such as
the additional count rates and ratios from the proposed method, it
is possible to calculate correction factors for the $R_{n}, \,
R_{nn}, \, R_{nnn}$ values. Designing calibration measurements
that analyze the PSD error would be beneficial for further
development of the applicability and accuracy of the proposed
system.


\section{Summary and Conclusions}
\label{sec:Conclusions}

The measurement system presented in this paper uses techniques
that, when combined together, create a novel way of characterizing
materials using both neutron and gamma ray multiplets. A study was
performed to investigate the feasibility of the method, and to
gain some insight on whether it is an applicable approach to
materials identification and accounting.

The measurement system is very accurate in discriminating the neutron pulses from the gamma ray
pulses. This capability is vital for the correct classification of
mixed particle multiplets. The initial tests have shown that the system can
acquire rates of n, $\gamma$, nn, n$\gamma$, $\gamma\gamma$, nnn,
nn$\gamma$, n$\gamma\gamma$ and $\gamma\gamma\gamma$ multiplets in
a relatively short measurement time (10-30 minutes) for a $^{252}$Cf source with
19500 fissions/s and a shielded MOX source. In the future, several
aspects will be further investigated, including the number and
configuration of detectors, and the possible utilization of
shielding to avoid cross talk between detectors.

It has been shown that a number of characteristic values can be
obtained and ratios of the measured values can be used to assess
the type of reactions causing the particle emission such as
fission versus ($\alpha$, n) reaction. The measured rates obtained
in the measurements are reasonable for the simple design of the current system. The data analysis
shows that the ratios between rates could also be utilized as an
indicator of source type and possibly also source isotope
composition, once the analysis is extended in detail
to the multiplicity theory.


\section{Acknowledgments}

The work of Andreas Enqvist was in part supported by the Swedish Radiation
Safety Authority. Part of the work was supported by the U.S. Department of
Energy Office of Nuclear Energy and the Advanced Fuel Cycle Initiative Safeguards Campaign. Idaho National Laboratory is operated for the U.S.
Department of Energy by Battelle Energy Alliance under DOE contract DE-AC07-05-ID14517.


% The Appendices part is started with the command \appendix;
% appendix sections are then done as normal sections
%\appendix


\begin{thebibliography}{00}
%\bibliographystyle{plain}

% \bibitem{label}
% Text of bibliographic item

% notes:
% \bibitem{label} \note

% subbibitems:
% \begin{subbibitems}{label}
% \bibitem{label1}
% \bibitem{label2}
% If there is a note, it should come last:
% \bibitem{label3} \note
% \end{subbibitems}

\bibitem{ensslin98}
{N. Ensslin, Application Guide to Neutron Multiplicity Counting.
Los Alamos Report LA-13422-M (1998).}

\bibitem{ENMC}
{J.E. Stewart, H.O. Menlove, D.R. Mayo, W.H. Geist, L.A. Carrillo, G.D. Herrera, Los Alamos
Report LA-13743-M (2000).}

\bibitem{Klein}
{H. Klein, S. Neumann, Nucl. Instr. Meth. A 476, (2002) 132.}

\bibitem{Xu}
{Y.Xu, T.J.Downar, S.Avdic, V.Protopopescu, S.A.Pozzi, Techniques
for neutron spectrum unfolding from pulse height distributions
measured with liquid scintillators, in:Proceedings of the
Institute of Nuclear Materials Management 47th Annual Meeting,
Nashville, TN, July 16�20, 2006.}

\bibitem{Reginatto}
{M.Reginatto, P.Goldhagen, S.Neumann, Nucl. Instr. and Meth. A,
476 (2002) 242.}

\bibitem{LSMC}
{K. Frame, W. Clay, T. Elmont, E. Esch, P. Karpius, D. MacArthur,
E. McKigney, P. Santi, M. Smith, J. Thron, R. Williams, Nucl.
Instr. Meth. A, 579 (2007) 192.}

\bibitem{Sher}
{R. Sher and S. Undermyer, "The detection of fissionable materials
by nondestructive means," American Nuclear Society (1980). }

\bibitem{NMIS}
{J.T. Mihalczo, J.A. Mullens, J.K. Mattingly, T.E. Valentine,
Nucl. Instr. Meth. A, 450 (2000) 531.}

\bibitem{Flaska}
{M. Flaska S.A. Pozzi, Nucl. Instr. Meth. A, 577 (2007) 654.}

\bibitem{PazPal08}
{I. P\'azsit and L. P\'al, ``Neutron Fluctuations - a Treatise on
the Physics of Branching Processes'', Elsevier Science Ltd,
London, New York, Tokyo, 2008.}

\bibitem{AEP}
{S. Avdic, A. Enqvist, I. P\'{a}zsit, ESARDA Bulletin Vol. 43
(2009) 21.}

\bibitem{EPA}
{A. Enqvist, I. P\'{a}zsit, S. Avdic,
doi:10.1016/j.nima.2010.01.022.}

\bibitem{CC}
{A. Enqvist, M. Flaska, S.A.Pozzi,  Nucl. Instr. and Meth. A 595
(2008) 426.}

\bibitem{Ispra}
{S.A. Pozzi, S.D. Clarke, M. Flaska, P. Peerani, Nucl. Instr. and
Meth. A, 608 (2009) 310.}

\bibitem{Adams}
{J.M Adams, G. White, Nucl. Instr. and Meth. 156 (1978) 459.}

\bibitem{Sperr}
{H. Sperr, H Spieler, M.R. Maier, D Evers, Nucl. Instr. and Meth.
116 (1974) 55.}

\bibitem{Valentine99}
{T. E. Valentine, ``Evaluation of prompt fission gamma rays for
use in simulating nuclear safeguard measurements,''
ORNL/TM-1999/300.}

\bibitem{Aniso}
{A. M. Gagarski, I. S. Guseva, V. E. Sokolov, G. V. Val�ski et al.
Bulletin of the Russian Academy of Sciences: Physics 72 No. 6
(2008) 773.}

%\bibitem{Carillo02}
%{H. Vega-Carrillo, E. Manzanares-Acuna, A. Becerra-Ferreiro, A.
%Carrillo-Nunez, Applied Radiation and Isotopes 57 (2002) 167.}


\end{thebibliography}

\end{document}
